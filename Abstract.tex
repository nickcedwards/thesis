\chapter*{Abstract}
This thesis presents measurements of the \ZZ\ production cross section in
proton-proton collisions at \sqrtseq{7} and at \sqrtseq{8} using data recorded
by the ATLAS experiment at the LHC in 2011 and 2012.  Events are selected which
are consistent with two \Z\ bosons decaying to electrons or muons. The \cx\ is
first measured in a fiducial phase-space corresponding closely to the detector
acceptance. For the 7~\tev\ measurement, this volume is defined by requring four
electrons or muons with \ptgt{7} and \modetalt{3.16}, with minumum separation
between any pair of leptons (electrons or muons) \deltaRlt{0.2}. The leptons
must form two \ossf\ pairs, each with invariant mass \sstooos. The fiducial \cx\
measure in a dataset corresponding to a luminosity of
\LumiPassGRLTwentyEleven~\ifb\ is \ZZSevenTeVFiducialCrossSectionZZLLLL. For the
8~\tev\ measurement, the fiducial volume is defined in the same way, except with
the pseudo-rapidity requirement tightened to \modetalt{2.7}. The fiducial \cx\
measured in a dataset corresponding to a luminosity of
\LumiPassGRLTwentyTwelve~\ifb\ is \ZZEightTeVFiducialCrossSectionZZLLLL.
Addionally, a fiducial \cx\ allowing of the \Z\ bosons to be off shell, is
measured at \sqrtseq{7} by allowing one of the lepton pairs to have \mllgtt.
This is found to be \ZZSevenTeVFiducialCrossSectionZZsLLLL.

These results are then used to derive the total cross section for \ZZ\
production with \Z\ bosons in the mass range 66$\GeV$ to 116$\GeV$, by
correcting for the acceptance of the fiducial phase-space and the \Zll\
branching fractions. The total \cx\ is measure to be
\ZZSevenTeVTotalCrossSection\ at \sqrtseq{7} and \ZZEightTeVTotalCrossSection\ at
\sqrtseq{8}, which is consistent with the Standard Model
expectation of \ZZSevenTeVTheoryTotalCrossSection\ at \sqrtseq{7} and \ZZEightTeVTheoryTotalCrossSection\ at
\sqrtseq{8}, calculated at the next-to-leading order in QCD.
%Limits on anomalous neutral triple gauge boson couplings are derived.
The differential \cx\ in bins of three kinematic variables is also
presented.

The differential event yield as a function of the transverse momentum of the
highest transeverse momentum \Z\ boson is used to set limits on the strength of
anomolous \ZZZ\ and \ZZg\ neutral triple gauge boson couplings, which are
forbidden in th \sm. 
%Limits are set separately with the \sqrtseq{7} and

%\sqrtseq{8} data. 
The limits obtained with the \sqrtseq{8} data are the most
constraining to date.
