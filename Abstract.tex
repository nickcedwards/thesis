\vspace{-20 mm}
\chapter*{Abstract}
\vspace{-3mm}
This thesis presents measurements of the \ZZ\ production cross section in
proton-proton collisions at \sqrtseq{7} and at \sqrtseq{8}, using data recorded
by the ATLAS experiment at the LHC in 2011 and 2012.  Events are selected which
are consistent with two \Z\ bosons decaying to electrons or muons. The \cx\ is
first measured in a fiducial \phasespace\ corresponding closely to the detector
acceptance. For the 7~\tev\ measurement, this \phasespace\ is defined by requiring four
electrons or muons with \ptgt{7} and \modetalt{3.16}, with a minimum separation
between any pair of leptons (electrons or muons) of \deltaRlt{0.2}. The leptons
must form two \ossf\ pairs, each with invariant mass \sstooos. The fiducial \cx\
times branching ratio to four electrons or muons
measured in a dataset corresponding to an integrated luminosity of
\LumiPassGRLTwentyEleven~\ifb\ is \ZZSevenTeVFiducialCrossSectionZZLLLL. For the
8~\tev\ measurement, the fiducial \phasespace\ is defined in the same way, except with
the pseudo-rapidity requirement tightened to \modetalt{2.7}. The fiducial \cx\
measured in a dataset corresponding to an integrated luminosity of
\LumiPassGRLTwentyTwelve~\ifb\ is \ZZEightTeVFiducialCrossSectionZZLLLL.
Additionally, a fiducial \cx\ allowing one of the \Z\ bosons to be off shell is
measured at \sqrtseq{7} by relaxing the mass requirement on one of the lepton pairs
to \mllgtt. This is found to be \ZZSevenTeVFiducialCrossSectionZZsLLLL.

These results are then used to derive the total cross section for \ZZ\
production with \Z\ bosons in the mass range 66$\GeV$ to 116$\GeV$, by
correcting for the acceptance of the fiducial \phasespace\ and the \Zll\
branching fractions. The total \cx\ is measured to be
\ZZSevenTeVTotalCrossSection\ at \sqrtseq{7} and \ZZEightTeVTotalCrossSection\
at \sqrtseq{8}, which is consistent with the Standard Model expectation of
\ZZSevenTeVTheoryTotalCrossSection\ at \sqrtseq{7} and
\ZZEightTeVTheoryTotalCrossSection\ at \sqrtseq{8}, calculated to
next-to-leading order in QCD.  The differential \cx\ in bins of three kinematic
variables is also presented.

The differential event yield as a function of the transverse momentum of the
highest transverse momentum \Z\ boson is used to set limits on the strength of
anomalous \ZZZ\ and \ZZg\ neutral triple gauge boson couplings, which are
forbidden in the \sm. 
%Limits are set separately with the \sqrtseq{7} and \sqrtseq{8} data. 
The limits obtained with the \sqrtseq{8} data are the most constraining to date.
