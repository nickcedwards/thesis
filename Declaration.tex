\chapter*{Declaration}
I declare that the results presented here are the product of my own work, unless
explicit references are given in the text to the work of others. This thesis is
the result of the research I carried out in the Experimental Particle Physics
group in the School of Physics \& Astronomy of the University of Glasgow between
September 2009 and March 2013. It has not been submitted for any other degree
at the University of Glasgow or any other institution.

\hfill Nicholas Edwards

\chapter*{Authors Contributions}

Due to the large nature of modern particle physics experiments, results are produced in a very collaborative
manner, and the work presented in this thesis relies on the work of many other
members of the ATLAS collaboration.
The main individual contributions of the author are outlined below:

\begin{itemize}

\item {\bf \chap{SCT}, SCT Temperature Monitoring:} The SCT temperature
monitoring study was carried out entirely by the author, under the guidance of
Saverio D'Auria and Steve McMahon, and involved writing software using
the existing the Oracle database. 
%The author implemented the automatic
%publication of the monitoring plots to the existing SCT calibration monitoring
%website.

\item {\bf \chap{ObjEventSelection}, Object and Event Selection:} The author played a
leading role in the development and optimisation of the selection requirements,
and the estimates of the selection efficiencies and systematics.

\item {\bf \chap{BackgroundEstimate}, Background Estimate:} The author contributed to the 
development of the data-driven background estimate method, in association with other members
of the ATLAS \ZZ\ working-group, and provided a cross-check of the final
results (the results presented in the chapter being entirely the work of the
author). The author produced the \mc\ based background estimates.

\item {\bf \chap{CrossSection}, Cross Section:} The author produced all results for
this chapter.

\item {\bf \chap{TGCLimits}, TGC Limits:} The \AfterBurner\ reweighting code and the TGC
limit setting methodology were developed by other members of the ATLAS
collaboration.
The 7~\tev\ TGC limit results were initially produced by
other members of the \ZZ\ working-group, using inputs from the author on expected
yields and systematics. The author subsequently cross-checked these results. For
the 8~\tev\ limits, the author carried out the bin optimisation study and
produced the limits.

\end{itemize}
