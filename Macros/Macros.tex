\usepackage{xspace}

\newcommand{\fig}[1]{Figure~\ref{fig:#1}}
\newcommand{\tab}[1]{Table~\ref{table:#1}}
\renewcommand{\sec}[1]{Section~\ref{sec:#1}}
\newcommand{\chap}[1]{Chapter~\ref{chap:#1}}
\newcommand{\intro}[1]{{\it #1}}
\newcommand{\timestenpower}[1]{\ensuremath{\times 10 ^{#1}}}

% |eta|> etc
\newcommand{\modetaeq}[1]{\ensuremath{|\eta|=#1}}
\newcommand{\modetagt}[1]{\ensuremath{|\eta|>#1}}
\newcommand{\modetalt}[1]{\ensuremath{|\eta|<#1}}
\newcommand{\modetabetween}[2]{\ensuremath{#1<|\eta|<#2}}

% |phi|> etc
\newcommand{\modphieq}[1]{\ensuremath{|\phi|=#1}}
\newcommand{\modphigt}[1]{\ensuremath{|\phi|>#1}}
\newcommand{\modphilt}[1]{\ensuremath{|\phi|<#1}}
\newcommand{\modphibetween}[2]{\ensuremath{#1<|\phi|<#2}}

% pt, et gt (GeV)
\newcommand{\pteq}[1]{\ensuremath{\pt =#1} \GeV}
\newcommand{\ptgt}[1]{\ensuremath{\pt >#1} \GeV}
\newcommand{\ptlt}[1]{\ensuremath{\pt <#1} \GeV}
\newcommand{\eteq}[1]{\ensuremath{\et =#1} \GeV}
\newcommand{\etgt}[1]{\ensuremath{\et >#1} \GeV}
\newcommand{\etlt}[1]{\ensuremath{\et <#1} \GeV}

% pt, et gt (MeV)
\newcommand{\pteqMeV}[1]{\ensuremath{\pt =#1} \MeV}
\newcommand{\ptgtMeV}[1]{\ensuremath{\pt >#1} \MeV}
\newcommand{\ptltMeV}[1]{\ensuremath{\pt <#1} \MeV}
\newcommand{\eteqMeV}[1]{\ensuremath{\et =#1} \MeV}
\newcommand{\etgtMeV}[1]{\ensuremath{\et >#1} \MeV}
\newcommand{\etltMeV}[1]{\ensuremath{\et <#1} \MeV}

% delta eta, delta ph, deltaR
\newcommand{\deltaetadeltaphi}[2]{\ensuremath{\Delta \eta \times \Delta \phi = #1 \times #2}}
\newcommand{\deltaR}{\ensuremath{\Delta R}}
\newcommand{\deltaRlt}[1]{\ensuremath{\Delta R < #1}}

\newcommand{\deltaetalt}[1]{\ensuremath{\Delta \eta < #1 }}
\newcommand{\deltaphilt}[1]{\ensuremath{\Delta \phi < #1 }}

\newcommand{\chisquared}{\ensuremath{\chi^{2}}}
\newcommand{\chisquaredndof}{\ensuremath{\chi^{2}/N_{\rm{dof}}}}


\newcommand{\instlumiunit}{\ensuremath{\rm{cm^{2}s^{-1}}}}
% Co-ordinates
\newcommand{\x}{\ensuremath{x}}
\newcommand{\y}{\ensuremath{y}}
\newcommand{\z}{\ensuremath{z}}
%\let\phi={\ensuremath{\phi}}
%\let\theta={\ensuremath{\theta}}
\newcommand{\R}{\ensuremath{R}}

% Quoting qauntities with errors
\newcommand{\errSym}[1]{\ensuremath{\pm} #1}
\newcommand{\errAsym}[2]{\ensuremath{^{+#1}_{-#2}}}
\newcommand{\measStatSyst}[3]{#1 #2 (stat) #3 (syst)}
\newcommand{\measStatSystSym}[3]{\measStatSystLumi{#1}{\errSym{#2}}{\errSym{#3}}}
\newcommand{\measStatSystLumi}[4]{#1 #2 (stat) #3 (syst) #4 (lumi) }
\newcommand{\measStatSystLumiSym}[4]{\measStatSystLumi{#1}{\errSym{#2}}{\errSym{#3}}{\errSym{#4}}}
\newcommand{\predErr}[2]{#1 #2}

% ZZ production
\newcommand{\qqZZ}{\ensuremath{qq\to\ZZ}}
\newcommand{\ggZZ}{\ensuremath{gg\to\ZZ}}
% ZZ channels
\newcommand{\ZZs}{\ensuremath{ZZ^{*}}}
\newcommand{\ZZllvv}{\ensuremath{\ZZ\to\ell^{-}\ell^{+}\nu\bar{\nu}}} 
\newcommand{\ZZeevv}{\ensuremath{\ZZ\to e^{-} e^{+}\nu\bar{\nu}}} 
\newcommand{\ZZmmvv}{\ensuremath{\ZZ\to \mu^{-} \mu^{+}\nu\bar{\nu}}} 
\newcommand{\ZZllll}{\ensuremath{\ZZ\to\ell\ell\ell'\ell'}}
\newcommand{\ZZsllll}{\ensuremath{\ZZs\to\ell\ell\ell'\ell'}}
\newcommand{\ZZlmlplmlp}{\ensuremath{\ZZ\to\ell^{-}\ell^{+}\ell^{-}\ell^{+}}}
\newcommand{\ZZeeee}{\ensuremath{\ZZ\to e^{+}e^{-}e^{+}e^{-}}}
\newcommand{\ZZeemm}{\ensuremath{\ZZ\to e^{+}e^{-}\mu^{+}\mu^{-}}}
\newcommand{\ZZmmmm}{\ensuremath{\ZZ\to\mu^{+}\mu^{-}\mu^{+}\mu^{-}}}
\newcommand{\eeee}{\ensuremath{e^{+}e^{-}e^{+}e^{-}}}
\newcommand{\eemm}{\ensuremath{e^{+}e^{-}\mu^{+}\mu^{-}}}
\newcommand{\mmmm}{\ensuremath{\mu^{+}\mu^{-}\mu^{+}\mu^{-}}}
\newcommand{\ZorgZorgllll}{\ensuremath{(\Zorgv)(\Zorgv)\to\ell\ell\ell'\ell'}}
\newcommand{\ZorgZorglplmlplm}{\ensuremath{(\Zorgv)(\Zorgv)\to\ell^{+}\ell^{-}\ell'^{+}\ell'^{-}}}

\newcommand{\qqZZllll}{\ensuremath{qq\to\ZZllll}}
\newcommand{\ggZZllll}{\ensuremath{gg\to\ZZllll}}

\newcommand{\mZZ}{\ensuremath{m^{\ZZ}}}
\newcommand{\ptZZ}{\ensuremath{\pt^{\ZZ}}}

%Compataibilty
\newcommand{\zzllll}{\ZZllll}

% Higgs
\newcommand{\HZZ}{\ensuremath{H\to\ZZ}}
\newcommand{\Hgg}{\ensuremath{H\to\gamma\gamma}}
\newcommand{\HWW}{\ensuremath{H\to WW}}


% llll and llvv
\newcommand{\llvv}{\ensuremath{\ell^{-}\ell^{+}\nu\bar{\nu}}}
\newcommand{\llll}{\ensuremath{\ell^{-}\ell^{+}\ell^{-}\ell^{+}}}
%\newcommand{\ZZ}{\ensuremath{ZZ}}

%TGC stuff
\newcommand{\ZZZ}{\ensuremath{ZZZ}}
\newcommand{\ZZg}{\ensuremath{ZZ\gamma}}

\newcommand{\sqrts}[1]{\ensuremath{\sqrt{s} = #1 } TeV}

% Lazy text
\newcommand{\ossf}{opposite-sign same-flavour}
\newcommand{\met}{missing transverse energy}
\newcommand{\sm}{Standard Model}
\newcommand{\brem}{brehmsstrahlung}
\newcommand{\mc}{Monte-Carlo}
\newcommand{\mcsim}{Monte-Carlo simulation}
\newcommand{\pu}{pile-up}
\newcommand{\cx}{cross section}

\newcommand{\partDF}{PDF}
\newcommand{\probDF}{p.d.f.}

\newcommand{\dzero}{\ensuremath{d_{0}}}
\newcommand{\dzerosig}{\ensuremath{\dzero / \sigma (\dzero )}}
\newcommand{\qoverp}{\ensuremath{q/p}}

\newcommand{\ztt} {$Z\ra\tau^{+}\tau^{-}$}

\newcommand{\sstooos}{\ensuremath{66 < m_{\ll} < 116} \gev}
\newcommand{\sstooosZ}{\ensuremath{66 < m_{\Z} < 116} \gev}
\newcommand{\mZgtt}{\ensuremath{m_{\Z}>20} \gev}


%Electron ID variables
\newcommand{\loose}{\texttt {Loose}}
\newcommand{\medium}{\texttt {Medium}}
\newcommand{\tight}{\texttt {Tight}}
\newcommand{\loosePP}{\texttt{Loose++}}
\newcommand{\mediumPP}{\texttt {Medium++}}
\newcommand{\tightPP}{\texttt {Tight++}}

% Muon ID stuff
\newcommand{\staco}{\texttt {STACO}}
\newcommand{\muonboy}{\texttt {MuonBoy}}
\newcommand{\muid}{\texttt {MUID}}
\newcommand{\CaloTrkMuID}{\texttt {CaloTrkMuID}}
\newcommand{\mutag}{\texttt {MuTag}}

%Shower shape
\newcommand{\Eoverp}{\ensuremath{E/p}}
\newcommand{\Rhad}{\ensuremath{R_{\rm had}}}
%\newcommand{\Retatwo}{\ensuremath{R_{\eta^{2}}}}
\newcommand{\wetatwo}{\ensuremath{w_{\eta{2}}}}
\newcommand{\Reta}{\ensuremath{R_{\eta}}}
\newcommand{\wstot}{\ensuremath{w_{\rm stot}}}
\newcommand{\Eratio}{\ensuremath{E_{\rm ratio}}}
\newcommand{\fthree}{\ensuremath{f_{3}}}

% Jets
\newcommand{\antikt}{anti-\ensuremath{k_{t}}}


% Generators
\newcommand{\ggtwoZZ}{{\sc gg}2{\sc zz}\xspace}
\newcommand{\amcatnlo}{aMC@NLO\xspace}
\newcommand{\sherpa} {{\sc Sherpa}}
\newcommand{\jimmy} {{\sc Jimmy}}
\newcommand{\ggtwozz} {{\sc gg2zz}}
\newcommand{\madgraph} {{\sc Madgraph}}
\newcommand{\powhegbox} {{\sc PowhegBox}}
\newcommand{\mcfm} {{\sc mcfm}}
\newcommand{\geant}{{\sc Geant4}\xspace}
\newcommand{\pythia}{{\sc Pythia}\xspace}
\newcommand{\pythiaB}{{\sc PythiaB}\xspace}
\newcommand{\mcatnlo}{MC@NLO\xspace}
\newcommand{\ggWW}{ggWW\xspace}
\newcommand{\alpgen}{{\sc Alpgen}\xspace}
\newcommand{\alpgenJimmy}{{\sc Alpgen/Jimmy}\xspace}
\newcommand{\herwig}{{\sc Herwig}\xspace}
\newcommand{\Jimmy}{Jimmy\xspace}
\newcommand{\BHO}{BHO\xspace}
\newcommand{\BosoMC}{BosoMC\xspace}
\newcommand{\ggtwoww}{{\tt gg2ww}\xspace}

%%%%%%%%%%%%%%%%%%%%%%%%%%%%%%%%%%%%%%%%%
% Stolen from atlasphysics.sty

\let\sst=\scriptscriptstyle % Needed for some newcommand{ (psi} and eta prime, etc.) (EE)
\chardef\letterchar=11
\chardef\otherchar=12
\chardef\eolinechar=5
%
% +--------------------------------------------------------------------+
% |                                                                                   
% |  additions for WZ note        
% |                                                                                   
% +--------------------------------------------------------------------+
%

\newcommand{\dR}{\ensuremath{\Delta R}}
%\newcommand{\WZ}{\ensuremath{WZ}}
\newcommand{\Z}{\ensuremath{\Zboson}}
\newcommand{\W}{\ensuremath{\Wpm}}
\newcommand{\WW}{\ensuremath{\Wplus\Wminus}\xspace}
\newcommand{\WZ}{\ensuremath{\Wpm\Zzero}\xspace}
\newcommand{\WZl}{\ensuremath{\Wpm\Zzero}$\rightarrow \ell^\pm \nu \ell^+\ell^- $\xspace}
\newcommand{\Wg}{\ensuremath{\Wpm\gamma}\xspace}
\newcommand{\Zg}{\ensuremath{\Zzero\gamma}\xspace}
\newcommand{\ZZ}{\ensuremath{\Zzero\Zzero}}
\newcommand{\Zorgv}{\ensuremath{\Zzero/\gamma^{\textstyle *}}\xspace}
\newcommand{\MV}{\ensuremath{M_{V}} \xspace }%
\newcommand{\Mll}{\ensuremath{M_{\ell\ell}} \xspace }%
\newcommand{\Mee}{\ensuremath{M_{ee}} \xspace }%
\newcommand{\Mmm}{\ensuremath{M_{\mu\mu}} \xspace }%
\newcommand{\MT}{\ensuremath{M_{T}} \xspace }%
\newcommand{\ns}{\ifmmode {\mathrm{\ ns}}\else
                   \textrm{ns}\fi}%
\newcommand{\pb}{\mbox{pb}}%  picobarns.
\newcommand{\fb}{\mbox{fb}}%  femtobarns.
%
% +--------------------------------------------------------------------+
% |                                                                    |
% |  Hours:minutes macro                                               |
% |                                                                    |
% +--------------------------------------------------------------------+
%
\newcount\hrs\newcount\minu\newcount\temptime
\newcommand{\hm}{\hrs=\time \divide\hrs by 60 \minu=\time\temptime=\hrs
\multiply\temptime by 60%
\advance\minu by -\temptime
\ifnum\minu<10 \let\zerofill=0\else \let\zerofill=\relax\fi
 \the\hrs:\zerofill\the\minu}
%
% +--------------------------------------------------------------------+
% |                                                                    |
% |  Useful symbols for use in or out of math mode                     |
% |                                                                    |
% +--------------------------------------------------------------------+
%
\newcommand{\ra}{\ensuremath{\rightarrow}}%  "GOES TO" arrow.
\newcommand{\la}{\ensuremath{\leftarrow}}%   "GETS" arrow.
\let\rarrow=\ra
\let\larrow=\la
\newcommand{\lapprox}{\ensuremath{\sim\kern-1em\raise 0.65ex\hbox{$<$}}}%  Or use \lsim
\newcommand{\rapprox}{\ensuremath{\sim\kern-1em\raise 0.65ex\hbox{$>$}}}%  and \rsim.
\newcommand{\gam}{\ensuremath{\gamma}}
\newcommand{\rts} {\ensuremath{\sqrt{s}}}
\newcommand{\stat}{\mbox{$\;$(stat.)}}
\newcommand{\syst}{\mbox{$\;$(syst.)}}
%
% +--------------------------------------------------------------------+
% |                                                                    |
% |  sin2thetaW m_W m_Z etc.                                           |
% |                                                                    |
% +--------------------------------------------------------------------+
%
\newcommand{\Mtau}{\ensuremath{m_{\tau}}}
\newcommand{\swsq}{\ensuremath{\sin^2\!\theta_{W}}}
\newcommand{\swel}{\ensuremath{\sin^2\!\theta_{\mathrm{eff}}^{\mathrm{lept}}}}
\newcommand{\swsqb}{\ensuremath{\sin^2\!\overline{\theta}_{W}}}
\newcommand{\swsqon}{\ensuremath{\swsq\equiv 1-\mW^2/\mZ^2}} % Lower-case masses (EE)
\newcommand{\gv}{\ensuremath{g_{\mathrm{V}}}} % Subscripts roman not italic (EE)
\newcommand{\ga}{\ensuremath{g_{\mathrm{A}}}} % Subscripts roman not italic (EE)
\newcommand{\gvbar}{\ensuremath{\bar{g}_\mathrm{V}}} % Subscripts roman not italic (EE)
\newcommand{\gabar}{\ensuremath{\bar{g}_\mathrm{A}}} % Subscripts roman not italic (EE)
%
% +--------------------------------------------------------------------+
% |                                                                    |
% |  Particle-antiparticle pair notations                              |
% |                                                                    |
% +--------------------------------------------------------------------+
%
\newcommand{\antibar}[1]{\ensuremath{#1\bar{#1}}}
\newcommand{\tbar}{\ensuremath{\bar{t}}}
\newcommand{\ttbar}{\antibar{t}}
\newcommand{\bbar}{\ensuremath{\bar{b}}}
\newcommand{\bbbar}{\antibar{b}}
\newcommand{\cbar}{\ensuremath{\bar{c}}}
\newcommand{\ccbar}{\antibar{c}}
\newcommand{\sbar}{\ensuremath{\bar{s}}}
\newcommand{\ssbar}{\antibar{s}}
\newcommand{\ubar}{\ensuremath{\bar{u}}}
\newcommand{\uubar}{\antibar{u}}
\newcommand{\dbar}{\ensuremath{\bar{d}}}
\newcommand{\ddbar}{\antibar{d}}
\newcommand{\fbar}{\ensuremath{\bar{f}}}
\newcommand{\ffbar}{\antibar{f}}
\newcommand{\qbar}{\ensuremath{\bar{q}}}
\newcommand{\qqbar}{\antibar{q}}
\newcommand{\nbar}{\ensuremath{\bar{\nu}}}
\newcommand{\nnbar}{\antibar{\nu}}
%
% +--------------------------------------------------------------------+
% |                                                                    |
% |  e+e-, etc.                                                        |
% |                                                                    |
% +--------------------------------------------------------------------+
%
\newcommand{\ee}{\ensuremath{e^+ e^-}}%
\newcommand{\epm}{\ensuremath{e^{\pm}}}%
\newcommand{\epem}{\ensuremath{e^+ e^-}}%
\newcommand{\mumu}{\ensuremath{\mathrm{\mu^+ \mu^-}}}%
\newcommand{\tautau}{\ensuremath{\mathrm{\tau^+ \tau^-}}}%
\let\muchless=\ll
\renewcommand{\ll}{\ensuremath{\ell^+ \ell^-}}%
\newcommand{\lnu}{\ensuremath{\ell \nu}}%
%
% +--------------------------------------------------------------------+
% |                                                                    |
% |  Useful Z0 type stuff    Gammas, asymmetries                       |
% |                                                                    |
% +--------------------------------------------------------------------+
\newcommand{\Zzero}{\ensuremath{Z}}
\newcommand{\Zboson}{\ensuremath{Z}}
\newcommand{\Wplus}{\ensuremath{W^+}}
\newcommand{\Wminus}{\ensuremath{W^-}}
\newcommand{\Wp}{\ensuremath{W^+}}
\newcommand{\Wm}{\ensuremath{W^-}}
\newcommand{\Wboson}{\ensuremath{W}}%
\newcommand{\Wpm}{\ensuremath{W^{\pm}}}%
\newcommand{\Wmp}{\ensuremath{W^{\mp}}}%
\newcommand{\Zzv}{\ensuremath{\Zzero^{\textstyle *}}}
\newcommand{\Abb}{\ensuremath{A_{\bbbar}}}
\newcommand{\Acc}{\ensuremath{A_{\ccbar}}}
\newcommand{\Aqq}{\ensuremath{A_{\qqbar}}}
\newcommand{\Afb}{\ensuremath{A_{{fb}}}} % Subscript italic not roman (EE)
\newcommand{\GZ}{\ensuremath{\Gamma_{Z}}}
\newcommand{\GW}{\ensuremath{\Gamma_{W}}}
\newcommand{\GH}{\ensuremath{\Gamma_{H}}}
\newcommand{\GamHad}{\ensuremath{\Gamma_{\mathrm{had}}}}
\newcommand{\Gbb}{\ensuremath{\Gamma_{\bbbar}}}
\newcommand{\Rbb}{\ensuremath{R_{\bbbar}}}
\newcommand{\Gcc}{\ensuremath{\Gamma_{\ccbar}}}
\newcommand{\Gvis}{\ensuremath{\Gamma_{\mathrm{vis}}}}
\newcommand{\Ginv}{\ensuremath{\Gamma_{\mathrm{inv}}}}
% +--------------------------------------------------------------------+
% |                                                                    |
% |  B-physics                                                         |
% |                                                                    |
% +--------------------------------------------------------------------+
%
\newcommand{\Bstar}{\ensuremath{B^{*}}}
\newcommand{\chic}{\ensuremath{\raise.4ex\hbox{$\chi$}_{{c}}}} % Raised & tightened, as chib (EE)
\newcommand{\BoBo}{\ensuremath{B^{0}\mbox{--}\bar{B}^{0}}} % en-dash not hyphen (EE)
\newcommand{\BodBod}{\ensuremath{B^{0}_{d}\mbox{--}\bar{B}^{0}_{d}}} % en-dash not hyphen (EE)
\newcommand{\BosBos}{\ensuremath{B^{0}_{s}\mbox{--}\bar{B}^{0}_{s}}} % en-dash not hyphen (EE)
\newcommand{\chib}{\ensuremath{\raise.4ex\hbox{$\chi$}_{{b}}}} % Bit of space removed (EE)
\newcommand{\Epsb}  {\ensuremath{\epsilon_{b}}} % Subscript italic not roman (EE)
\newcommand{\Epsc}  {\ensuremath{\epsilon_{c}}} % Subscript italic not roman (EE)
\newcommand{\Kstar}    {\ensuremath{K^{*}}}
\newcommand{\Dstar}   {\ensuremath{D^{*}}} % Italic not roman (EE)
\newcommand{\Dsstar}   {\ensuremath{D^{**}}} % Italic not roman (EE)
\newcommand{\etpt}     {\ensuremath{1/p_{\mathrm{T}} - 1/E_{\mathrm{T}}}} 
    % p not P, and subscripts roman not italic (EE)
\newcommand{\etptsig}  {\ensuremath{(1/p_{\mathrm{T}} - 1/E_{\mathrm{T}})/(\sigma(1/p_{\mathrm{T}}))}}  
    % p not P, and subscripts roman not italic (EE)
\newcommand{\Bd} {\ensuremath{B_d^0}}
\newcommand{\Bs} {\ensuremath{B_s^0}}
\newcommand{\Bu} {\ensuremath{B_u}}
\newcommand{\Bc} {\ensuremath{B_c}}
\newcommand{\Lb} {\ensuremath{\Lambda_b}}
\newcommand{\btol} {\ensuremath{b \rightarrow \ell}} % Italic not roman (EE)
\newcommand{\ctol} {\ensuremath{c \rightarrow \ell}} % Italic not roman (EE)
\newcommand{\btoctol} {\ensuremath{b \rightarrow c \rightarrow \ell}} % Italic not roman (EE)
%
% +--------------------------------------------------------------------+
% |                                                                    |
% |  J/psi, psi prime, etc.                                            |
% |                                                                    |
% +--------------------------------------------------------------------+
%
\let\psii=\psi  %  Save normal "\psi" newcommand{, since} I renewcommand{ it.}
\renewcommand{\psi}{\ensuremath{\psii}}%
\newcommand{\jpsi}{\ensuremath{J/\psi}}
\newcommand{\Jee}{\ensuremath{\Jpsi\ra\epem}}
\newcommand{\Jmm}{\ensuremath{\Jpsi\ra\mumu}}
\newcommand{\Jmumu}{\ensuremath{\Jpsi\ra\mumu}}
\newcommand{\Brjl}{\ensuremath{\mathrm{Br}(\Jpsi \ra \ll)}} % Italic not roman (EE)
\newcommand{\psip}{\ensuremath{\psi^{\sst\prime}}}

%
% +--------------------------------------------------------------------+
% |                                                                    |
% |  QCD (Simplified all of these: no hbox, etc.) (EE)                 |
% |                                                                    |
% +--------------------------------------------------------------------+
%
\newcommand{\alphas}{\ensuremath{\alpha_{\mathrm{S}}}} % Subscript roman not italic (EE)
\newcommand{\NF}{\ensuremath{N_{\mathrm{F}}}}
\newcommand{\NC}{\ensuremath{N_{\mathrm{C}}}}
\newcommand{\CF}{\ensuremath{C_{\mathrm{F}}}}
\newcommand{\CA}{\ensuremath{C_{\mathrm{A}}}}
\newcommand{\TF}{\ensuremath{T_{\mathrm{F}}}}
\newcommand{\Lms}{\ensuremath{\Lambda_{\overline{\mathrm{MS}}}}}
\newcommand{\Lmsfive}{\ensuremath{\Lambda^{(5)}_{\overline{\mathrm{MS}}}}}
\newcommand{\KT}{\ensuremath{k_{\perp}}}                                   
%
% +--------------------------------------------------------------------+
% |                                                                    |
% |  CKM matrix                                                        |
% |                                                                    |
% +--------------------------------------------------------------------+
%
\newcommand{\Vcb}{\ensuremath{\vert V_{cb} \vert}}
\newcommand{\Vub}{\ensuremath{\vert V_{ub} \vert}}
\newcommand{\Vtd}{\ensuremath{\vert V_{td} \vert}}
\newcommand{\Vts}{\ensuremath{\vert V_{ts} \vert}}
\newcommand{\Vtb}{\ensuremath{\vert V_{tb} \vert}}
\newcommand{\Vcs}{\ensuremath{\vert V_{cs} \vert}}
\newcommand{\Vud}{\ensuremath{\vert V_{ud} \vert}}
\newcommand{\Vus}{\ensuremath{\vert V_{us} \vert}}
\newcommand{\Vcd}{\ensuremath{\vert V_{cd} \vert}}
%
% +--------------------------------------------------------------------+
% |                                                                    |
% |  New particle stuff                                                |
% |                                                                    |
% +--------------------------------------------------------------------+
%
\newcommand{\Azero}{\ensuremath{A^0}}%
\newcommand{\hzero}{\ensuremath{h^0}}%
\newcommand{\Hzero}{\ensuremath{H^0}}%
\newcommand{\Hboson}{\ensuremath{H}}%
\newcommand{\Hplus}{\ensuremath{H^+}}%
\newcommand{\Hminus}{\ensuremath{H^-}}%
\newcommand{\Hpm}{\ensuremath{H^{\pm}}}%
\newcommand{\Hmp}{\ensuremath{H^{\mp}}}%
\newcommand{\susy}[1]{\ensuremath{\tilde{#1}}}%
\newcommand{\ellell}{\ensuremath{\mathrm{\ell^+ \ell^-}}}%
\newcommand{\ggino}{\ensuremath{\mathchoice%
      {\displaystyle\raise.4ex\hbox{$\displaystyle\tilde\chi$}}%
         {\textstyle\raise.4ex\hbox{$\textstyle\tilde\chi$}}%
       {\scriptstyle\raise.3ex\hbox{$\scriptstyle\tilde\chi$}}%
 {\scriptscriptstyle\raise.3ex\hbox{$\scriptscriptstyle\tilde\chi$}}}}

\newcommand{\chinop}{\ensuremath{\mathchoice%
      {\displaystyle\raise.4ex\hbox{$\displaystyle\tilde\chi^+$}}%
         {\textstyle\raise.4ex\hbox{$\textstyle\tilde\chi^+$}}%
       {\scriptstyle\raise.3ex\hbox{$\scriptstyle\tilde\chi^+$}}%
 {\scriptscriptstyle\raise.3ex\hbox{$\scriptscriptstyle\tilde\chi^+$}}}}
\newcommand{\chinom}{\ensuremath{\mathchoice%
      {\displaystyle\raise.4ex\hbox{$\displaystyle\tilde\chi^-$}}%
         {\textstyle\raise.4ex\hbox{$\textstyle\tilde\chi^-$}}%
       {\scriptstyle\raise.3ex\hbox{$\scriptstyle\tilde\chi^-$}}%
 {\scriptscriptstyle\raise.3ex\hbox{$\scriptscriptstyle\tilde\chi^-$}}}}
\newcommand{\chinopm}{\ensuremath{\mathchoice%
      {\displaystyle\raise.4ex\hbox{$\displaystyle\tilde\chi^\pm$}}%
         {\textstyle\raise.4ex\hbox{$\textstyle\tilde\chi^\pm$}}%
       {\scriptstyle\raise.3ex\hbox{$\scriptstyle\tilde\chi^\pm$}}%
 {\scriptscriptstyle\raise.3ex\hbox{$\scriptscriptstyle\tilde\chi^\pm$}}}}
\newcommand{\chinomp}{\ensuremath{\mathchoice%
      {\displaystyle\raise.4ex\hbox{$\displaystyle\tilde\chi^\mp$}}%
         {\textstyle\raise.4ex\hbox{$\textstyle\tilde\chi^\mp$}}%
       {\scriptstyle\raise.3ex\hbox{$\scriptstyle\tilde\chi^\mp$}}%
 {\scriptscriptstyle\raise.3ex\hbox{$\scriptscriptstyle\tilde\chi^\mp$}}}}

\newcommand{\chinoonep}{\ensuremath{\mathchoice%
      {\displaystyle\raise.4ex\hbox{$\displaystyle\tilde\chi^+_1$}}%
         {\textstyle\raise.4ex\hbox{$\textstyle\tilde\chi^+_1$}}%
       {\scriptstyle\raise.3ex\hbox{$\scriptstyle\tilde\chi^+_1$}}%
 {\scriptscriptstyle\raise.3ex\hbox{$\scriptscriptstyle\tilde\chi^+_1$}}}}
\newcommand{\chinoonem}{\ensuremath{\mathchoice%
      {\displaystyle\raise.4ex\hbox{$\displaystyle\tilde\chi^-_1$}}%
         {\textstyle\raise.4ex\hbox{$\textstyle\tilde\chi^-_1$}}%
       {\scriptstyle\raise.3ex\hbox{$\scriptstyle\tilde\chi^-_1$}}%
 {\scriptscriptstyle\raise.3ex\hbox{$\scriptscriptstyle\tilde\chi^-_1$}}}}
\newcommand{\chinoonepm}{\ensuremath{\mathchoice%
      {\displaystyle\raise.4ex\hbox{$\displaystyle\tilde\chi^\pm_1$}}%
         {\textstyle\raise.4ex\hbox{$\textstyle\tilde\chi^\pm_1$}}%
       {\scriptstyle\raise.3ex\hbox{$\scriptstyle\tilde\chi^\pm_1$}}%
 {\scriptscriptstyle\raise.3ex\hbox{$\scriptscriptstyle\tilde\chi^\pm_1$}}}}

\newcommand{\chinotwop}{\ensuremath{\mathchoice%
      {\displaystyle\raise.4ex\hbox{$\displaystyle\tilde\chi^+_2$}}%
         {\textstyle\raise.4ex\hbox{$\textstyle\tilde\chi^+_2$}}%
       {\scriptstyle\raise.3ex\hbox{$\scriptstyle\tilde\chi^+_2$}}%
 {\scriptscriptstyle\raise.3ex\hbox{$\scriptscriptstyle\tilde\chi^+_2$}}}}
\newcommand{\chinotwom}{\ensuremath{\mathchoice%
      {\displaystyle\raise.4ex\hbox{$\displaystyle\tilde\chi^-_2$}}%
         {\textstyle\raise.4ex\hbox{$\textstyle\tilde\chi^-_2$}}%
       {\scriptstyle\raise.3ex\hbox{$\scriptstyle\tilde\chi^-_2$}}%
 {\scriptscriptstyle\raise.3ex\hbox{$\scriptscriptstyle\tilde\chi^-_2$}}}}
\newcommand{\chinotwopm}{\ensuremath{\mathchoice%
      {\displaystyle\raise.4ex\hbox{$\displaystyle\tilde\chi^\pm_2$}}%
         {\textstyle\raise.4ex\hbox{$\textstyle\tilde\chi^\pm_2$}}%
       {\scriptstyle\raise.3ex\hbox{$\scriptstyle\tilde\chi^\pm_2$}}%
 {\scriptscriptstyle\raise.3ex\hbox{$\scriptscriptstyle\tilde\chi^\pm_2$}}}}

\newcommand{\nino}{\ensuremath{\mathchoice%
      {\displaystyle\raise.4ex\hbox{$\displaystyle\tilde\chi^0$}}%
         {\textstyle\raise.4ex\hbox{$\textstyle\tilde\chi^0$}}%
       {\scriptstyle\raise.3ex\hbox{$\scriptstyle\tilde\chi^0$}}%
 {\scriptscriptstyle\raise.3ex\hbox{$\scriptscriptstyle\tilde\chi^0$}}}}

\newcommand{\ninoone}{\ensuremath{\mathchoice%
      {\displaystyle\raise.4ex\hbox{$\displaystyle\tilde\chi^0_1$}}%
         {\textstyle\raise.4ex\hbox{$\textstyle\tilde\chi^0_1$}}%
       {\scriptstyle\raise.3ex\hbox{$\scriptstyle\tilde\chi^0_1$}}%
 {\scriptscriptstyle\raise.3ex\hbox{$\scriptscriptstyle\tilde\chi^0_1$}}}}
\newcommand{\ninotwo}{\ensuremath{\mathchoice%
      {\displaystyle\raise.4ex\hbox{$\displaystyle\tilde\chi^0_2$}}%
         {\textstyle\raise.4ex\hbox{$\textstyle\tilde\chi^0_2$}}%
       {\scriptstyle\raise.3ex\hbox{$\scriptstyle\tilde\chi^0_2$}}%
 {\scriptscriptstyle\raise.3ex\hbox{$\scriptscriptstyle\tilde\chi^0_2$}}}}
\newcommand{\ninothree}{\ensuremath{\mathchoice%
      {\displaystyle\raise.4ex\hbox{$\displaystyle\tilde\chi^0_3$}}%
         {\textstyle\raise.4ex\hbox{$\textstyle\tilde\chi^0_3$}}%
       {\scriptstyle\raise.3ex\hbox{$\scriptstyle\tilde\chi^0_3$}}%
 {\scriptscriptstyle\raise.3ex\hbox{$\scriptscriptstyle\tilde\chi^0_3$}}}}
\newcommand{\ninofour}{\ensuremath{\mathchoice%
      {\displaystyle\raise.4ex\hbox{$\displaystyle\tilde\chi^0_4$}}%
         {\textstyle\raise.4ex\hbox{$\textstyle\tilde\chi^0_4$}}%
       {\scriptstyle\raise.3ex\hbox{$\scriptstyle\tilde\chi^0_4$}}%
 {\scriptscriptstyle\raise.3ex\hbox{$\scriptscriptstyle\tilde\chi^0_4$}}}}

\newcommand{\gravino}{\ensuremath{\tilde{G}}}%
\newcommand{\Zprime}{\ensuremath{Z^\prime}}
\newcommand{\Zstar}{\ensuremath{Z^{*}}}
\newcommand{\squark}{\ensuremath{\tilde{q}}}
\newcommand{\squarkL}{\ensuremath{\tilde{q}_{\mathrm{L}}}} % Subscript roman not italic (EE)
\newcommand{\squarkR}{\ensuremath{\tilde{q}_{\mathrm{R}}}} % Subscript roman not italic (EE)
\newcommand{\gluino}{\ensuremath{\tilde{g}}}
\renewcommand{\stop}{\ensuremath{\tilde{t}}}
\newcommand{\stopone}{\ensuremath{\tilde{t}_1}}
\newcommand{\stoptwo}{\ensuremath{\tilde{t}_2}}
\newcommand{\stopL}{\ensuremath{\tilde{t}_{\mathrm{L}}}} % Subscript roman not italic (EE)
\newcommand{\stopR}{\ensuremath{\tilde{t}_{\mathrm{R}}}} % Subscript roman not italic (EE)
\newcommand{\sbottom}{\ensuremath{\tilde{b}}}
\newcommand{\sbottomone}{\ensuremath{\tilde{b}_1}}
\newcommand{\sbottomtwo}{\ensuremath{\tilde{b}_2}}
\newcommand{\sbottomL}{\ensuremath{\tilde{b}_{\mathrm{L}}}} % Subscript roman not italic (EE)
\newcommand{\sbottomR}{\ensuremath{\tilde{b}_{\mathrm{R}}}} % Subscript roman not italic (EE)
\newcommand{\slepton}{\ensuremath{\tilde{\ell}}}
\newcommand{\sleptonL}{\ensuremath{\tilde{\ell}_{\mathrm{L}}}} % Subscript roman not italic (EE)
\newcommand{\sleptonR}{\ensuremath{\tilde{\ell}_{\mathrm{R}}}} % Subscript roman not italic (EE)
\newcommand{\sel}{\ensuremath{\tilde{e}}}
\newcommand{\selL}{\ensuremath{\tilde{e}_{\mathrm{L}}}} % Subscript roman not italic (EE)
\newcommand{\selR}{\ensuremath{\tilde{e}_{\mathrm{R}}}} % Subscript roman not italic (EE)
\newcommand{\smu}{\ensuremath{\tilde{\mu}}}
\newcommand{\smuL}{\ensuremath{\tilde{\mu}_{\mathrm{L}}}} % Subscript roman not italic (EE)
\newcommand{\smuR}{\ensuremath{\tilde{\mu}_{\mathrm{R}}}} % Subscript roman not italic (EE)
\newcommand{\stau}{\ensuremath{\tilde{\tau}}}
\newcommand{\stauL}{\ensuremath{\tilde{\tau}_{\mathrm{L}}}} % Subscript roman not italic (EE)
\newcommand{\stauR}{\ensuremath{\tilde{\tau}_{\mathrm{R}}}} % Subscript roman not italic (EE)
\newcommand{\stauone}{\ensuremath{\tilde{\tau}_1}}
\newcommand{\stautwo}{\ensuremath{\tilde{\tau}_2}}
\newcommand{\snu}{\ensuremath{\tilde{\nu}}}
%
% +--------------------------------------------------------------------+
% |                                                                    |
% |  pi, pi0, pi+, pi-, pi+-, eta, eta1, etc.                          |
% |                                                                    |
% +--------------------------------------------------------------------+
%
\let\pii=\pi
\renewcommand{\pi}{\ensuremath{\pii}}%
\newcommand{\pizero}{\ensuremath{\pii^0}}%
\newcommand{\piplus}{\ensuremath{\pii^+}}%
\newcommand{\piminus}{\ensuremath{\pii^-}}%
\newcommand{\pipm}{\ensuremath{\pii^{\pm}}}%
\newcommand{\pimp}{\ensuremath{\pii^{\mp}}}%
\let\etaa=\eta
%\newcommand{\eta}{\ensuremath{\etaa}}%
\newcommand{\etaprime}{\ensuremath{\eta^{\sst\prime}}}%
%
% +--------------------------------------------------------------------+
% |                                                                    |
% |  K0, K+, K-, K0L, K0S                                              |
% |                                                                    |
% +--------------------------------------------------------------------+
%
\newcommand{\kzero}{\ensuremath{K^0}}%
\newcommand{\kzerobar}{\ensuremath{\overline{K}\vphantom{K}^0}}%
%
\newcommand{\kaon}{\ensuremath{K}}%
\newcommand{\kplus}{\ensuremath{K^+}}%
\newcommand{\kminus}{\ensuremath{K^-}}%
\newcommand{\kzeroL}{\ensuremath{K^0_{\mathrm{L}}}} % Subscript roman not italic (EE)
\newcommand{\kzerol}{\ensuremath{K^0_{\mathrm{L}}}} % Subscript roman not italic (EE)
\newcommand{\klong}{\ensuremath{K^0_{\mathrm{L}}}} % Subscript roman not italic (EE)
\newcommand{\kzeroS}{\ensuremath{K^0_{\mathrm{S}}}} % Subscript roman not italic (EE)
\newcommand{\kzeros}{\ensuremath{K^0_{\mathrm{S}}}} % Subscript roman not italic (EE)
\newcommand{\kshort}{\ensuremath{K^0_{\mathrm{S}}}} % Subscript roman not italic (EE)
%
% +--------------------------------------------------------------------+
% |                                                                    |
% |  Upsilons of various sorts                                         |
% |                                                                    |
% +--------------------------------------------------------------------+
%
\newcommand{\Ups}{\ensuremath{\mit\Upsilon}} % Should be italic (EE)
\newcommand{\Upsp}{\ensuremath{\mit\Upsilon^{\sst\prime}}} % Should be italic (EE)
\newcommand{\Upspp}{\ensuremath{\mit\Upsilon^{\sst\prime\prime}}} % Should be italic (EE)
\newcommand{\Upsppp}{\ensuremath{\mit\Upsilon^{\sst\prime\prime\prime}}} % Should be italic (EE)
\newcommand{\Upspppp}{\ensuremath{\mit\Upsilon^{\sst\prime\prime\prime\prime}}} % Should be italic (EE)
\newcommand{\itUpsp}{\ensuremath{\mit\Upsilon^{\sst\prime}}}%
\newcommand{\UoneS}{\ensuremath{\Upsilon(\mathrm{1S})}}%

%
% +--------------------------------------------------------------------+
% |                                                                    |
% |  Things like \ups4 --> Y(4S)                                       |
% |                                                                    |
% +--------------------------------------------------------------------+
%
\newcommand{\ups}[1]{\ensuremath{\mit{\Upsilon}(\mathrm{#1S})}} % Italic, fixed and simplified (EE)
%
% +--------------------------------------------------------------------+
% |                                                                    |
% |  Notation for the P-lines: \nspj211 -->  2 1P1, etc.               |
% |                                                                    |
% +--------------------------------------------------------------------+
%
\newcommand{\nsPj}[3]{\ensuremath{#1\,^{#2}\!P_{#3}}}%
\let\nspj=\nsPj
\newcommand{\nsSj}[3]{\ensuremath{#1\,^{#2}\!S_{#3}}}%
\let\nssj=\nsSj
%
% +--------------------------------------------------------------------+
% |                                                                    |
% |  Useful things for proton-proton physics                           |
% |                                                                    |
% +--------------------------------------------------------------------+
%
\newcommand{\E}{\ensuremath{E}}
\newcommand{\e}{\ensuremath{E}}
\newcommand{\pt}{\ensuremath{p_{\mathrm{T}}}} % Subscript roman not italic (EE)
\newcommand{\pT}{\ensuremath{p_{\mathrm{T}}}} % Subscript roman not italic (EE)
\newcommand{\et}{\ensuremath{E_{\mathrm{T}}}} % Subscript roman not italic (EE)
\newcommand{\eT}{\ensuremath{E_{\mathrm{T}}}} % Subscript roman not italic (EE)
\newcommand{\ET}{\ensuremath{E_{\mathrm{T}}}} % Subscript roman not italic (EE)
\newcommand{\HT}{\ensuremath{H_{\mathrm{T}}}} % Subscript roman not italic (EE)

\newcommand{\ptsq}{\ensuremath{p^2_{\mathrm{T}}}} % Fixed so it works correctly (EE)
\newcommand{\missET} {$E_{\mathrm{T}}^{\mathrm{miss}}$}
\newcommand{\etmiss} {\missET}

\newcommand{\micro}{\ensuremath{\rm{\mu}}}

\renewcommand{\b}{\ensuremath{b}}

% Single Zs, J/psi
\newcommand{\JPsiee}{\ensuremath{\jpsi\to e^{-} e^{+}}} 

\newcommand{\degr}{\ensuremath{^\circ}} % Removed mbox - caused problems and not needed (EE)
\newcommand{\abseta}{\ensuremath{|\eta|}}
\newcommand{\mh}{\ensuremath{m_h}}
\newcommand{\mW}{\ensuremath{m_W}}
\newcommand{\mZ}{\ensuremath{m_Z}}
\newcommand{\mZPDG}{\ensuremath{m_{Z}^{\rm PDG}}}
\newcommand{\mH}{\ensuremath{m_H}}
\newcommand{\mA}{\ensuremath{m_A}}
%\newcommand{\MET}{\missET}
%\newcommand{\met}{\missET}
\newcommand{\Wjj}{\ensuremath{W \rightarrow jj}}
\newcommand{\tjjb}{\ensuremath{t \rightarrow jjb}}
\newcommand{\Hbb}{\ensuremath{H \rightarrow b\bar b}}
\newcommand{\Zmm}{\ensuremath{Z \rightarrow \mu\mu}}
\newcommand{\Zee}{\ensuremath{Z \rightarrow ee}}
\newcommand{\Zll}{\ensuremath{Z \rightarrow \ell\ell}}
\newcommand{\Wln}{\ensuremath{W \rightarrow \ell\nu}}
\newcommand{\Wen}{\ensuremath{W \rightarrow e\nu}}
\newcommand{\Wmn}{\ensuremath{W \rightarrow \mu\nu}}
\newcommand{\Hllll}{\ensuremath{H \rightarrow ZZ^{(*)} \rightarrow \mu\mu\mu\mu}}
\newcommand{\Hmmmm}{\ensuremath{H \rightarrow \mu\mu\mu\mu}}
\newcommand{\Heeee}{\ensuremath{H \rightarrow eeee}}
\newcommand{\Amm}{\ensuremath{A \rightarrow \mu\mu}}
\newcommand{\Ztau}{\ensuremath{Z \rightarrow \tau\tau}}
\newcommand{\Wtau}{\ensuremath{W \rightarrow \tau\nu}}
\newcommand{\Atau}{\ensuremath{A \rightarrow \tau\tau}}
\newcommand{\Htau}{\ensuremath{H \rightarrow \tau\tau}}
\newcommand{\begL}{10$^{31}$~cm$^{-2}$~s$^{-1}$}
\newcommand{\lowL}{10$^{33}$~cm$^{-2}$~s$^{-1}$}
\newcommand{\highL}{10$^{34}$~cm$^{-2}$~s$^{-1}$}
\newcommand{\EjetRec}{\ensuremath{E_{\mathrm{rec}}}} % Subscript roman not italic (EE)
\newcommand{\PjetRec}{\ensuremath{p_{\mathrm{rec}}}} % Subscript roman not italic (EE)
\newcommand{\EjetTru}{\ensuremath{E_{\mathrm{truth}}}} % Subscript roman not italic (EE)
\newcommand{\PjetTru}{\ensuremath{p_{\mathrm{truth}}}} % Subscript roman not italic (EE)
\newcommand{\EjetDM}{\ensuremath{E_{\mathrm{DM}}}} % Subscript roman not italic (EE)
\newcommand{\Rcone}{\ensuremath{R_{\mathrm{cone}}}} % Subscript roman not italic (EE)
%
% +--------------------------------------------------------------------+
% |                                                                    |
% |  Some useful units                                                 |
% |                                                                    |
% +--------------------------------------------------------------------+
%
\newcommand{\TeV}{\ifmmode {\mathrm{\ Te\kern -0.1em V}}\else
                   \textrm{Te\kern -0.1em V}\fi}%
\newcommand{\GeV}{\ifmmode {\mathrm{\ Ge\kern -0.1em V}}\else
                   \textrm{Ge\kern -0.1em V}\fi}%
\newcommand{\MeV}{\ifmmode {\mathrm{\ Me\kern -0.1em V}}\else
                   \textrm{Me\kern -0.1em V}\fi}%
\newcommand{\keV}{\ifmmode {\mathrm{\ ke\kern -0.1em V}}\else
                   \textrm{ke\kern -0.1em V}\fi}%
\newcommand{\eV}{\ifmmode  {\mathrm{\ e\kern -0.1em V}}\else
                   \textrm{e\kern -0.1em V}\fi}%
\let\tev=\TeV
\let\gev=\GeV
\let\mev=\MeV
\let\kev=\keV
\let\ev=\eV

\newcommand{\TeVc}{\ifmmode {\mathrm{\ Te\kern -0.1em V}/c}\else
                   {\textrm{Te\kern -0.1em V}/$c$}\fi}%
\newcommand{\GeVc}{\ifmmode {\mathrm{\ Ge\kern -0.1em V}/c}\else
                   {\textrm{Ge\kern -0.1em V}/$c$}\fi}%
\newcommand{\MeVc}{\ifmmode {\mathrm{\ Me\kern -0.1em V}/c}\else
                   {\textrm{Me\kern -0.1em V}/$c$}\fi}%
\newcommand{\keVc}{\ifmmode {\mathrm{\ ke\kern -0.1em V}/c}\else
                   {\textrm{ke\kern -0.1em V}/$c$}\fi}%
\newcommand{\eVc}{\ifmmode  {\mathrm{\ e\kern -0.1em V}/c}\else
                   {\textrm{e\kern -0.1em V}/$c$}\fi}%
\let\tevc=\TeVc
\let\gevc=\GeVc
\let\mevc=\MeVc
\let\kevc=\keVc
\let\evc=\eVc

\newcommand{\TeVcc}{\ifmmode {\mathrm{\ Te\kern -0.1em V}/c^2}\else
                   {\textrm{Te\kern -0.1em V}/$c^2$}\fi}%
\newcommand{\GeVcc}{\ifmmode {\mathrm{\ Ge\kern -0.1em V}/c^2}\else
                   {\textrm{Ge\kern -0.1em V}/$c^2$}\fi}%
\newcommand{\MeVcc}{\ifmmode {\mathrm{\ Me\kern -0.1em V}/c^2}\else
                   {\textrm{Me\kern -0.1em V}/$c^2$}\fi}%
\newcommand{\keVcc}{\ifmmode {\mathrm{\ ke\kern -0.1em V}/c^2}\else
                   {\textrm{ke\kern -0.1em V}/$c^2$}\fi}%
\newcommand{\eVcc}{\ifmmode  {\mathrm{\ e\kern -0.1em V}/c^2}\else
                   {\textrm{e\kern -0.1em V}/$c^2$}\fi}%
\let\tevcc=\TeVcc
\let\gevcc=\GeVcc
\let\mevcc=\MeVcc
\let\kevcc=\keVcc
\let\evcc=\eVcc

\newcommand{\cm}{\ifmmode  {\mathrm{\ cm}}\else
                   \textrm{~cm}\fi}%
%
\newcommand{\ifb}{\mbox{fb$^{-1}$}}%  Inverse femtobarns.
\newcommand{\ipb}{\mbox{pb$^{-1}$}}%  Inverse picobarns.
\newcommand{\inb}{\mbox{nb$^{-1}$}}%  Inverse nanobarns.
%
\newcommand{\mass}[1]{\ensuremath{m_{#1#1}}}%  "\mass{\mu}" produces "msub{mumu}".
\newcommand{\twomass}[2]{\ensuremath{m_{#1#2}}}% 
%
\newcommand{\Ecm}{\ensuremath{E_{\mathrm{cm}}}} % Subscript roman not italic (EE)
%
% +--------------------------------------------------------------------+
% |                                                                    |
% |  "Box-squared" operator, as in Klein-Gordon. Command is "\boxsq".  |
% |                                                                    |
% +--------------------------------------------------------------------+
%
\newbox\boxsqbox
\newdimen\boxsize\boxsize=1.2ex%
\newcommand{\boxop}{%
\setbox\boxsqbox=\vbox{\hrule depth0.8pt width0.8\boxsize height0pt%
                       \kern0.8\boxsize
                       \hrule height0.8pt width0.8\boxsize depth0pt}%
           \hbox{%
           \vrule height1.0\boxsize width0.8pt depth0pt%
           \copy\boxsqbox
           \vrule height1.0\boxsize width0.8pt depth0pt\kern1.5pt}}%
\newcommand{\boxsq}{\ensuremath{\boxop^2}}%
% +--------------------------------------------------------------------+
% |                                                                    |
% |  Theoretical notations                                             |
% |                                                                    |
% +--------------------------------------------------------------------+
%
\newcommand{\spinor}[1]{\ensuremath{\left(\matrix{#1_1\cr#1_2\cr#1_3\cr#1_4\cr}\right)}} % Math mode (EE)
%\newcommand{\pmb}[1]{\setbox0=\hbox{$#1$}%  This is "poor man's boldface".
%  \kern-.025em\copy0\kern-1.0\wd0%
%  \kern.05em\copy0\kern-1.0\wd0%
%  \kern-.025em\raise.0433em\box0}%
\newcommand{\grad}{\pmb{\nabla}}%
%
% +--------------------------------------------------------------------+
% |                                                                    |
% |  The decay symbol, to be used in \eqalign.                         |
% |  It works like: \[\eqalign{a\ra &b+c\cr &\dk &e+f\cr &&\dk g+h}\]  |
% |                                                                    |
% |                  a  -->  b + c                                     |
% |                          |                                         |
% |                          |                                         |
% |                          +----> e + f                              |
% |                                 |                                  |
% |                                 |                                  |
% |                                 +----> g + h                       |
% |                                                                    |
% +--------------------------------------------------------------------+
%
\newdimen\dkwidth
\newcommand{\dk}{%
   \dkwidth=\baselineskip
   {\newcommand{\to}{\rightarrow}%  allows "\rightarrowfill" to work.
   \kern 3pt%
   \hbox{%
      \raise 3pt%
      \hbox{%
         \vrule height 0.8\dkwidth width 0.7pt depth0pt%
      }%
      \kern-0.4pt%
      \hbox to 1.5\dkwidth{%
         \rightarrowfill
      }%
   \kern0.6em%
   }}%
}%
%
% +--------------------------------------------------------------------+
% |                                                                    |
% |  Renewcommand{ \eqalign} to allow more than one column; very             |
% |  useful for multiple decays as newcommand{ above.}                      |
% |                                                                    |
% +--------------------------------------------------------------------+
%
%\unlock
\newcommand{\eqalign}[1]{%
   \,
   \vcenter{%
      \openup\jot\m@th
      \ialign{%
         \strut\hfil$\displaystyle{##}$&&$%
         \displaystyle{{}##}$\hfil\crcr#1\crcr%
      }%
   }%
   \,
}%
%\lock
%
% +--------------------------------------------------------------------+
% |                                                                    |
% |  JOURNALS (for MISC newcommand{, see} also ../biblio/ATLASstyle.bst)|
% |                                                                    |
% +--------------------------------------------------------------------+
%
\newcommand {\AcPA}   {Acta Phys. Austriaca{} }
\newcommand {\ARevNS} {Ann.{} Rev.{} Nucl.{} Sci.{} }
\newcommand {\CPC}    {Comp.{} Phys.{} Comm.{} }
\newcommand {\FortP}  {Fortschr.{} Phys.{} }
\newcommand {\IJMP}   {Int.{} J.{} Mod.{} Phys.{} }
\newcommand {\JETP}   {Sov.{} Phys.{} JETP{} }
\newcommand {\JETPL}  {JETP Lett.{} }
\newcommand {\JaFi}   {Jad.{} Fiz.{} }
\newcommand {\JMP}    {J.{} Math.{} Phys.{} }
\newcommand {\MPL}    {Mod.{} Phys.{} Lett.{} }
\newcommand {\NCim}   {Nuovo Cimento{} }
\newcommand {\NIM}    {Nucl.{} Instrum.{} Meth.{} }
\newcommand {\NP}     {Nucl.{} Phys.{} }
\newcommand {\PL}     {Phys.{} Lett.{} }
\newcommand {\PR}     {Phys.{} Rev.{} }
\newcommand {\PRL}    {Phys.{} Rev.{} Lett.{} }
\newcommand {\PRep}   {Phys.{} Rep.{} }   
\newcommand {\RMP}    {Rev.{} Mod.{} Phys.{} }
\newcommand {\ZfP}    {Z.{} Phys.{} }
\newcommand {\EPJ}    {Eur.{} Phys.{} J.{} }

%%%%%%%%%%%%%%%%%%%%%%%%%%%%%%%%%%%%%%%%%
