\graphicspath{{Chapters/Introduction/Figures/}}

\chapter*{Introduction}
\addcontentsline{toc}{chapter}{Introduction}
\label{chap:Introduction}

The LHC began high energy proton proton collisions at in 2010, and since then
has accumulated high statistics datasets at centre of mass energies of
and  \sqrtseq{7}, and from 2012, \sqrtseq{8}. These high statistics samples
allow for precision measurements of the \sm, as well as direct searches for
physics beyond the \sm. Of particular interest are the diboson production
procceses, where gauge bosons, the force-carriers of the \sm, are produced in
pairs. Such processes are relatively rare, occuring only roughly 1 in every
$XX\timestenpower{??}$ collisions. The diboson production rates are sensitive to
contributions from new physics, where heavy new particles could decay to diboson
pairs. Non-resonant diboson production is also a major background process to
seraches for, and studies of the Higgs boson, which until recently was a key
missing component of the \sm. Recently teh ATLAS and CMS experiments both
reported the discovery of a new particle with mass around 125~\gev, with
properties consistent with the Higgs. The good undertsanding of non-resonant
diboson production was a key ingredient to this discovery. This thesis is
concerned with diboson \ZZ\ prodction, where the neurtral \Z\ boson is produced
in pairs.

Measuring diboson proce

This thesis is divided into three parts. Firstly, \refpart{bg} gives the theoretical background
and motivation to the work presented in this thesis.
Secondly, \refpart{experiment} describes the ATLAS detector setupand software.

Finally, \refpart{analysis} described the \ZZ\ analysis.

