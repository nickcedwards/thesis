\graphicspath{{Chapters/Introduction/Figures/}}

\chapter*{Introduction}
\addcontentsline{toc}{chapter}{Introduction}
\label{chap:Introduction}

The LHC began high energy proton proton collisions at in 2010, and since then
has accumulated high statistics datasets at centre of mass energies of
and  \sqrtseq{7}, and from 2012, \sqrtseq{8}. These high statistics samples
allow for precision measurements of the \sm, as well as direct searches for
physics beyond the \sm. Of particular interest are the diboson production
procceses, where gauge bosons, the force-carriers of the \sm, are produced in
pairs. Such processes are relatively rare, occuring only roughly 1 in every
$XX\timestenpower{??}$ collisions. The diboson production rates are sensitive to
contributions from new physics, where heavy new particles could decay to diboson
pairs. Non-resonant diboson production is also a major background process to
seraches for, and studies of the Higgs boson, which until recently was a key
missing component of the \sm. Recently teh ATLAS and CMS experiments both
reported the discovery of a new particle with mass around 125~\gev, with
properties consistent with the Higgs. The good understanding of non-resonant
diboson production was a key ingredient to this discovery. This thesis is
concerned with diboson \ZZ\ prodction, where the neurtral \Z\ boson is produced
in pairs.

Measuring diboson processes also gives a direct probe of the gauge structure of
the \ew\ sector of the \sm. In the \sm, the so called \intro{neutral triple gauge
couplings (\TGC s)} \ZZZ\ and \ZZg\ are forbidden, and diboson production may
only precede by combinations of $Zqq$ vertices. Anomolous \ZZZ\ and \ZZg\ couplings are
predicted in some moels of physics beyond the \sm, and would lead to an
increased production rate of diboson events, particularly when the diboson
system is produced with high energy. By studying the rate of \ZZ\ production and
the kinematic distributions of the observed events, one can search for anomolous
\TGC s, and in the absence of any evidence for them, set limits on the maximum
size of the couplings.

This thesis is divided into three parts. Firstly, \refpart{bg} gives the theoretical background
and motivation to the work presented in this thesis. \chap{Theory} gives an
introduction to the \sm, with particular emphasis on the \ew\ theory, and gives
an introduction to the phenomoneology of particle collisions at the LHC and the
computational tools available for simulating them. ~\chap{TheoryZZ} gives a more
detailed introduction to \ZZ\ production, \TGC s, and previous experimental
results.

Secondly, \refpart{experiment} describes the experimental setup.
\chap{Detector} gives an outline of the LHC and it's experiments, and a
description of the ATLAS detector. \chap{SCT} describes work done by the author
monitoring the performance of the cooling system of the Semiconductor Tracker,
part of the ATLAS inner detector. An overview of the particle reconstruction and
identification algorithms and their performance is given in
\chap{Reconstruction}.

Thirdly and finally, \refpart{analysis} describes the main topic of this thesis, the \ZZ\
analysis. \chap{ObjEventSelection} describes requirements used to select \ZZ\
events and reject background events. \chap{BackgroundEstimate} describes
estimates of the background processes to the selection requriements.
\chap{CrossSection} presents the observed events passing the selection
requirements, their kinematic distributions, and measurements of the \ZZ\ \cx.
These events are then used to set limits on \TGC s, which is described
in~\chap{TGCLimits}.

