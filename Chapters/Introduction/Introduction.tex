\graphicspath{{Chapters/Introduction/Figures/}}

\chapter*{Introduction}
\addcontentsline{toc}{chapter}{Introduction}
\label{chap:Introduction}

The LHC began high energy proton-proton collisions in 2010, and since then
has accumulated high statistics datasets at centre of mass energies of
\sqrtseq{7}, and from 2012, \sqrtseq{8}. These high statistics samples
allow for precision measurements of the \sm, as well as direct searches for
physics beyond the \sm. 

Of particular interest are the diboson production processes, where gauge bosons,
the force-carriers of the \sm, are produced in pairs. Such processes are
relatively rare, occurring as rarely as one in every $10^{13}$
collisions. Diboson production rates are sensitive to contributions from new
physics, where heavy new particles could decay to diboson pairs, and so such
processes provide ideal search channels for new particles. Non-resonant diboson
production is also a major background process to searches for, and studies of,
the Higgs boson, which until recently was a major missing component of the \sm.
Recently, the ATLAS and CMS experiments both reported the discovery of a new
particle with mass around 125~\gev, with properties consistent with the Higgs.
The good understanding of non-resonant diboson production was a key ingredient
to this discovery. This thesis is concerned with diboson \ZZ\ production, where
the neutral \Z\ boson is produced in pairs.

Measuring diboson processes also gives a direct probe of the gauge structure of
the \ew\ sector of the \sm. In the \sm, the so called \intro{neutral triple gauge
couplings (\TGC s)} \ZZZ\ and \ZZg\ are forbidden, and diboson production may
only precede by combinations of $Zq\bar{q}$ vertices. Anomalous \ZZZ\ and \ZZg\ couplings are
predicted in some models of physics beyond the \sm, and would lead to an
increased production rate of diboson events, particularly when the diboson
system is produced with high energy. By studying the rate of \ZZ\ production and
the kinematic distributions of the observed events, one can search for anomalous
\TGC s, and in the absence of any evidence for them, set limits on the maximum
size of the couplings.

This thesis is divided into three parts. Firstly, \refpart{bg} gives the theoretical background
and motivation to the work presented in this thesis. \chap{Theory} gives an
introduction to the \sm, with particular emphasis on the \ew\ theory, and gives
an introduction to the phenomenology of particle collisions at the LHC and the
computational tools available for simulating them. ~\chap{TheoryZZProduction} gives a more
detailed introduction to \ZZ\ production, \TGC s, and previous experimental
results.

Secondly, \refpart{experiment} describes the experimental setup.
\chap{Detector} gives an outline of the LHC and its experiments, and a
description of the ATLAS detector. \chap{SCT} describes work done by the author
monitoring the performance of the cooling system of the Semiconductor Tracker
(SCT), part of the ATLAS inner detector. An overview of the particle reconstruction and
identification algorithms and their performance is given in
\chap{Reconstruction}.

Thirdly, \refpart{analysis} describes the main topic of this thesis, the
measurement of the \ZZ\ production \cx\ and limits on \TGC s. 
\chap{ObjEventSelection} describes requirements used to select \ZZ\
events and reject background processes. \chap{BackgroundEstimate} describes
estimates of the size of the contribution from background processes.
\chap{CrossSection} presents the observed events passing the selection
requirements, their kinematic distributions, and measurements of the \ZZ\ \cx.
Separate measurements of the \cx\ at \sqrtseq{7} and \sqrtseq{8} are made. These
are described `side by side', with small differences in the methodology between
the two measurements described as appropriate. At each centre-of-mass energy,
two measurements of the \cx\ are provided: firstly a measurement in a restricted
fiducial \phasespace\ corresponding closely to the experimental kinematic
selections, termed the \intro{fiducial \cx}, and secondly a \intro{total} \cx,
correcting for the acceptance of the fiducial \phasespace.
These events are then used to set limits on \TGC s, which is described
in~\chap{TGCLimits}.

Many of the results presented in this thesis contributed to ATLAS conference
notes and publications. An initial measurement of the \ZZ\ \cx\ at \sqrtseq{7}
and limits on \TGC s using the four-lepton final state in 1~\ifb\ of data was
presented in \cite{ATLAS-CONF-2011-107} and later published in
\cite{ATLAS_ZZ4l:1fb2011}. An updated measurement of the \cx\ using the full
7~\tev\ dataset of 4.7~\ifb\ was given in~\cite{ATLAS-CONF-2012-026}, and later
a paper published~\cite{ATLAS:2012kg} giving combined results with an analysis
of \ZZllvv\ decays, including \TGC\ limits and differential \cx s.
A first measurement of the \ZZ\ \cx\ at \sqrtseq{8} \cx\ using the first
5.8~\ifb\ of the 2012 dataset was given in~\cite{ATLAS-CONF-2012-090}, and an
updated measurement using the full 2012 dataset of 20~\ifb\ given in~\cite{ATLAS-CONF-2013-020}.

