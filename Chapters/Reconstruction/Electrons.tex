Electrons are reconstructed in the central region (\modetalt{2.5}) by a
combination of a calorimeter cluster and an inner detector track; this is
referred to below as the `standard' electron algorithm. Reconstruction begins
with the construction of seed clusters in the calorimeter. These are formed from
calorimeter towers of size \deltaetadeltaphi{0.025}{0.025}, corresonding to the
size of cells in the second layer of the electromagnetic calorimeter. This
results in a grid of $200 \times 256$ towers. The energy
of the tower is the sum of all the EM calorimeter cells in all three layers
falling within the tower. Where cells are shared between more than one tower the
energy is shared according the the fractional overlap of the cell with each
tower. The seed cells are formed by sliding a
window of size \deltaetadeltaphi{0.075}{0.125} ($3 \times 5$ towers) over the
grid of towers and identifying local maxima with \etgt{2.5} \gev. The position
of the cluster is taken to be the energy weighted $\eta$, $\phi$ barycentre of
cells in a window around the centre of the cluster. If two seeds are closer than
\deltaetadeltaphi{0.050}{0.050} only the one with higher \et\ is kept.

Electron canidadtes are then formed by matching an Inner Detector track to the
calorimeter cells. If there is a reasonable agreement (0.2 in $\eta$ and 0.1 in
$\phi$) between a track's co-ordinates measured at the origin of the track and
the cluster seed, the track is extrapolated from its last measurement point to
the middle layer of the EM calorimeter (TRT-only tracks are automatically
extrapolated). The track is matched to the cluster if it has $\Delta \eta <
0.05$ and $\Delta \phi < 0.1$ on the side where thre track bends and  $\Delta
\phi < 0.05$ on the opposite side. This assymetry in the $\phi$ window is to
account for the fact that the electrons undergo heavy energy losses from
bremmstrahlung due to the large amount of material in the Inner Detector,
particularly at high $\eta$. If a seed cluster matches to at least one track,
and electron candidate is formed. Seed clusters with no track matches are
considered as photon candidates. If several tracks match, tracks with hits with
silicon hits are preferred, and the one with the smallest \deltaR to the cluser
is chosen. In the case of TRT-only tracks, only a matching in $\phi$ is required, 
due to the limited $\eta$ resolution in the TRT.

The clusters for electron canidates are then rebuilt, using a fixed rectangle of size 
\deltaetadeltaphi{0.075}{0.175} ($0.125 \times 0.125$) in the barrel (endcap),
again using a sliding window to find the local maximum. The cluster energy is
the sum of four components: the estimated energy desposits before the
calorimeter, the measured energy in the cluster, the estimated leakage laterally
into other calorimeter cells and the estimated longitudinal leakage behind the
EM calorimeter. The four terms are paramaterised as a function of the measured
cluster energy in the presampler (where it exists) and in each of the three
calorimeter layers, based on detailed simulations of energy depositions in the
calorimeters and the dead material. At this stage additional callibrations are
applied to the electron energy based on measurements of \Zee\ and
\JPsiee~\cite{Aad:2011mk}.

%The cluster reconstruction efficiency is very efficent for electrons:

\subsection{Electron triggers}
\label{sec:reco-el-triggers}
