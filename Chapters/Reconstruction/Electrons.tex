\subsection{Trigger}
\label{sec:reco-el-triggers}

\subsection{Reconstruction}

Electrons are reconstructed in the central region (\modetalt{2.5}) by searching
for EM calorimeter clusters and matching them to Inner Detector tracks; this is
referred to as the `standard' electron algorithm and is described below.
In the forward regions (\modetabetween{2.5}{4.9}) there is no Inner Detector
tracking, and electrons are reconstructed solely from calorimeter clusters; this
is also described below. The `standard' algorithm drops in efficiency at very
low \pt\
(a few \gev), but an alternative `soft-e' algorithm can be used to recover
efficiency. This uses Inner Detector tracks as seeds, which it extrapolates into
the EM calorimeter and attempts to build clusters. It is not used in
this thesis so is not described in detail here.

\subsubsection{Standard Electron Reconstruction}

Reconstruction begins
with the construction of seed clusters in the EM calorimeter. These are formed from
calorimeter towers of size \deltaetadeltaphi{0.025}{0.025}, corresponding to the
size of cells in the second layer of the EM calorimeter. This
results in a grid of $200 \times 256$ towers. The energy
of the tower is the sum of the cells in all three layers
falling within the tower. Where cells are shared between more than one tower, the
energy is shared according the fractional overlap of the cell with each
tower. The seed cells are formed by sliding a
window of size \deltaetadeltaphi{0.075}{0.125} ($3 \times 5$ towers) over the
grid of towers and identifying local maxima with \etgt{2.5}. The position
of the cluster is taken to be the energy weighted $\eta$, $\phi$ barycentre of
cells in a window around the centre of the cluster. If two seeds are closer than
\deltaetadeltaphi{0.050}{0.050} only the one with higher \et\ is kept.

Electron candidates are then formed by matching Inner Detector tracks to the
seed clusters. If there is a reasonable agreement ($\Delta \eta <0.2$ and
$\Delta \phi <0.1$) between a track's co-ordinates (measured at the origin of the
track) and a cluster seed, the track is extrapolated from its last measurement point to
the middle layer of the EM calorimeter (TRT-only tracks are automatically
extrapolated). In $\eta$, the cluster is required to be within 
of$\Delta \eta <0.05$ of the track. In $\phi$, the cluster must be within $\Delta \phi < 0.1$ of
the track if it falls on the side towards which the track bends, or $\Delta
\phi < 0.05$ if it is on the opposite side. This asymmetry in the $\phi$
requirement is to
account for the fact that the electrons undergo heavy energy losses from
bremsstrahlung, due to the large amount of material in the Inner Detector, which
will tend to increase their bending,
particularly at high $\eta$. If a seed cluster matches to at least one track,
and electron candidate is formed. Seed clusters with no track matches are
considered as photon candidates. If several tracks match, tracks with hits with
silicon hits are preferred, and the one closest to the cluster in \deltaR\ 
is chosen. In the case of TRT-only tracks, only a matching in $\phi$ is required, 
due to the limited $\eta$ resolution in the TRT.

The clusters for electron candidates are then rebuilt, using a fixed rectangle of size 
\deltaetadeltaphi{0.075}{0.175} ($0.125 \times 0.125$) in the barrel (endcap),
again using a sliding window to find the local maximum. The cluster energy is
the sum of four components: the estimated energy deposits before the
calorimeter, the measured energy in the cluster, the estimated leakage laterally
into other calorimeter cells and the estimated longitudinal leakage behind the
EM calorimeter. The four terms are parameterised as a function of the measured
cluster energy in the pre-sampler (where it exists) and the measured energy in each of the three
calorimeter layers, based on detailed simulations of energy depositions in the
calorimeters and the dead material. At this stage additional calibrations are
applied to the electron energy based on measurements of \Zee\ and
\JPsiee~\cite{Aad:2011mk}. The energy of the electron is taken as the cluster
energy, and the direction as the track $\eta$ and $\phi$, providing the track
has sufficient silicon hits.

\subsubsection{Forward Electron Reconstruction}

Forward electrons are reconstructed from energy deposits in the calorimeter
only. `Topological clusters' are formed by grouping neighbouring cells in three
dimensions. The topological clusters do not have a fixed size, but will depend
on the energy deposit and the clustering criteria used. Cells with a
signal versus noise significance above a high threshold $t_{\rm{seed}}$, are used as seeds.
Neighbouring cells with a signal significance above a lower threshold
$t_{\rm{cell}}$ are added to the cluster. The neighbours may act as
secondary seeds if they have signal significance above an intermediate threshold
$t_{\rm{neighbour}}$. For electron topological clusters $t_{\rm{seed}} =
t_{\rm{neighbour}}$. The lower threshold at the cell perimeter ensures that
tails of showers are not discarded, while the higher thresholds for seeds and
neighbors suppress electronics and pile-up noise. The cells are split if they
contain more than one local maxima above a certain energy threshold. 

An electron candidate is constructed if the cluster has \etgt{5} and only a
small hadronic energy component. The energy
of the electron is taken as the sum of the energy of all cells belonging to the cluster,
corrected for energy loss before the calorimeter and lateral and longitudinal
leakages. The direction of the forward electron is taken as the barycentre of the cells
belonging to the cluster.

\subsection{Identification}

%The cluster reconstruction efficiency is very efficient for electrons:
