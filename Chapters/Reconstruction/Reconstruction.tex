\graphicspath{{Chapters/Reconstruction/Figures/}}
\chapter{Object Reconstruction}

In this chapter the software algorithms used to reconstruct and identify physics
objects are described. Reconstruction involves reconstructing tracks in the
Inner Detector and Muon Spectrometer, idenitfying interaction vertices,
and identifying clusters of energy deposits in the calorimeter systems. These are
then combined to reconstruct physics `objects' such as electrons, muons, jets, photons,
tau leptons, as well as to measure properties as the event such as missing
transverse energy. 
Since triggering on electrons and muons is a crucial component of the
measurements described in this thesis, they are described in more detail here.

\label{chap:Reconstruction}

\section{Tracking}

Particles traversing the Inner detector travel in an approximately helical path
under the influence of the magnetic field, leaving hits in the various detector
components that they traverse 

\label{sec:reco-tracking}
\section{Vertex Finding}
\label{sec:reco-vertexing}
\section{Cluster Reconstruction}
\label{sec:reco-clustering}
\section{Electron Reconstruction and Identification}
\label{sec:reco-el}
\subsection{Electron triggers}
\label{sec:reco-triggers}
\section{Muon Reconstruction and Identification}
\label{sec:reco-mu}
\subsection{Muon triggers}
\label{sec:reco-triggers}
\section{Muon reconstruction}
