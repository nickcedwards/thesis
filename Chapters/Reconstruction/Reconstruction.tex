\graphicspath{{Chapters/Reconstruction/Figures/}}
\chapter{Object Reconstruction}

In this chapter the software algorithms used to reconstruct and identify physics
objects are described. Reconstruction involves reconstructing tracks in the
Inner Detector and Muon Spectrometer, idenitfying interaction vertices,
and identifying clusters of energy deposits in the calorimeter systems. These are
then combined to reconstruct physics `objects' such as electrons, muons, jets, photons,
tau leptons, as well as to measure properties as the event such as missing
transverse energy. 
Since triggering on electrons and muons is a crucial component of the
measurements described in this thesis, they are described in more detail here.

\label{chap:Reconstruction}

\section{Tracking}

Particles traversing the Inner Detector travel in an approximately helical path
under the influence of the magnetic field, leaving hits in the various detector
components that they traverse. In order to use these hits to identify and
measure particles, it is necessary to reconstruct particle tracks from these
hits, in a process known as tracking. At the collision energies and levels of
pileup at the LHC, there will typically be hundreds of hits in the Inner
Detector. The tracking algorithm must be able to correctly associate hits with
tracks, as well as reconstruct the track parameters. As well as interacting with
the active elements of the detector, particles will also interact with the
material in the Inner Detector, leading to multiple scatterings, ionisation
energy loss and, especially for electrons, radiational energy loss from
bremsstrahlung. A detailed descirption of the ATLAS tracking is given
in~\cite{1742-6596-119-3-032014}.

A particle's trajectory can be described by five parameters, $\bf{x_{i}}$ . In ATLAS, the
parameters are chosen to be:

\begin{equation}
{\bf x_{i}} = (l_{1}, l_{2}, \phi, \theta, q/p)
\end{equation}

where $l_{1}$, $l_{2}$ are two co-ordinates in the frame of the detector surface
of the measurent and the other three parameters describe the momentum of the track
in the global frame.

\subsection{Inside Out Tracking}

In ATLAS, the main tracking algorithm is known as `inside-out' tracking, as it
begin in the inner layers of the detector and work outwards. The first step is
the formation of space-points from the measurements in the silicon detectors. In
the Pixel detector a space-point corresponds simply to a hit in the detector. In
the SCT space-points are required to have hits in both sides of the module in
order to give a measurement in $z$ (due to the stereo-angle). Track seeds are
then formed from combinations of space-points in the three Pixel detector layers
and the first layer of the SCT. 

These seeds are used to build roads through the
rest of the detector elements. A Kalman fitter-smoother~\cite{Fruhwirth:1987fm} is used to follow the
trajectory, succesively adding hits to the track fit. From a given layer, the
Kalman filter will predict the track parameters on the next detector layer, then
update the track parameters and covariances taking into account the measurements
found on the next layer (filtering), as well as refining the estimates for the
track parameters on the previous layers based on the new measurement. The Kalman
filter takes into account linear distortions to the track from multiple
scattering and from ionisation energy loss. Energy loss through brehmsstrahlung
is, however, highly non-gaussian, and is not modelled well in this approach. Roughly
10\% of seeds will lead to track candidates.

The next step in the procedure is ambiguity resolution; many of the track
candidates found in the track finding will share hits, will be as a result of
fakes, or will erroneaously incoporporate outliers. At this stage the track is
refitted with a global \chisquared\ fit using a refined reconstruction geometry
with more detailed material description and omitting outliers. 
A score is assigned to each track, based upon the fit quality \chisquaredndof,
the number of hits on the track, the prescence of overlapping hits on a layer,
and with
penalties for `holes' (missing hits). The scores for different detector elements
are weighted giving greater weight to more precise detector elements. Ambiguties
are resolved by choosing the track with the greater score; tracks with a score
below a certain threshold are discarded.

The track is then extended into the TRT, associating drift circles from the
straw tubes with the track. The extended tracks are refitted once again, using
the full information of all three detectors. The quality of the extended track
is compared to the quality of the silicon only track; the track extension is
kept only if it improves the quality of the fit.

\subsection{Outside In Tracking}

\label{sec:reco-tracking}
\section{Vertex Finding}
\label{sec:reco-vertexing}
\section{Cluster Reconstruction}
\label{sec:reco-clustering}
\section{Electron Reconstruction and Identification}
\label{sec:reco-el}
\subsection{Electron triggers}
\label{sec:reco-triggers}
\section{Muon Reconstruction and Identification}
\label{sec:reco-mu}
\subsection{Muon triggers}
\label{sec:reco-triggers}
\section{Muon reconstruction}
