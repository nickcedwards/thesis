%\graphicspath{{Chapters/Reconstruction/Figures/}}
\graphicspath{{Figures/}}

\chapter{Object Reconstruction}
\label{chap:Reconstruction}

In this chapter the software algorithms used to reconstruct and identify physics
objects are described. Reconstruction involves reconstructing tracks in the
Inner Detector and Muon Spectrometer, identifying interaction vertices,
and identifying clusters of energy deposits in the calorimeter systems. These are
then combined to reconstruct physics `objects' such as electrons, muons, jets, photons,
tau leptons, as well as to measure properties as the event such as missing
transverse energy. 
Since triggering on electrons and muons is a crucial component of the
measurements described in this thesis, they are described in more detail here.

\section{Tracking}
\label{sec:reco-tracking}

Particles traversing the Inner Detector travel in an approximately helical path
under the influence of the magnetic field, leaving hits in the various detector
components that they traverse. In order to use these hits to identify and
measure particles, it is necessary to reconstruct particle tracks from these
hits, in a process known as tracking. At the collision energies and levels of
pileup at the LHC, there will typically be hundreds of hits in the Inner
Detector. The tracking algorithm must be able to correctly associate hits with
tracks, as well as reconstruct the track parameters. As well as interacting with
the active elements of the detector, particles will also interact with the
material in the Inner Detector, leading to multiple scatterings, ionisation
energy loss and, especially for electrons, radiation energy loss from
bremsstrahlung. A detailed description of the ATLAS tracking is given
in~\cite{1742-6596-119-3-032014}.

A particle's trajectory can be described by five parameters, $\bf{x_{i}}$ . In ATLAS, the
parameters are chosen to be:

\begin{equation}
{\bf x_{i}} = (l_{1}, l_{2}, \phi, \theta, q/p)
\end{equation}

where $l_{1}$, $l_{2}$ are two co-ordinates in the frame of the detector surface
of the measurement and the other three parameters describe the momentum of the track
in the global frame.

\subsection{Inside-Out Tracking}

In ATLAS, the main tracking algorithm is known as `inside-out' tracking, as it
begin in the inner layers of the detector and work outwards. The first step is
the formation of space-points from the measurements in the silicon detectors. In
the Pixel detector a space-point corresponds simply to a hit in the detector. In
the SCT space-points are required to have hits in both sides of the module in
order to give a measurement in $z$ (due to the stereo-angle). Track seeds are
then formed from combinations of space-points in the three Pixel detector layers
and the first layer of the SCT. 

These seeds are used to build roads through the
rest of the detector elements. A Kalman fitter-smoother~\cite{Fruhwirth:1987fm} is used to follow the
trajectory, successively adding hits to the track fit. From a given layer, the
Kalman filter will predict the track parameters on the next detector layer, then
update the track parameters and covariances taking into account the measurements
found on the next layer (filtering), as well as refining the estimates for the
track parameters on the previous layers based on the new measurement. The Kalman
filter takes into account linear distortions to the track from multiple
scattering and from ionisation energy loss. Energy loss through brehmsstrahlung
is, however, highly non-gaussian, and is not modelled well in this approach. Roughly
10\% of seeds will lead to track candidates.

The next step in the procedure is ambiguity resolution; many of the track
candidates found in the track finding will share hits, or will be as a result of
fakes, or will erroneously incorporate outliers. At this stage the track is
refitted with a global \chisquared\ fit~\cite{1742-6596-119-3-032013}, using a refined reconstruction geometry
with more detailed material description and omitting outliers. 
A score is assigned to each track, based upon the fit quality \chisquaredndof,
the number of hits on the track, the presence of overlapping hits on a layer,
and with
penalties for `holes' (missing hits). The scores for different detector elements
are weighted giving greater weight to more precise detector elements. Ambiguities
are resolved by choosing the track with the greater score; tracks with a score
below a certain threshold are discarded.

The track is then extended into the TRT, associating drift circles from the
straw tubes with the track. The extended tracks are refitted once again, using
the full information of all three detectors. The quality of the extended track
is compared to the quality of the silicon only track; the track extension is
kept only if it improves the quality of the fit.

\subsection{Outside-In Tracking}

The inside-out tracking procedure will fail to find tracks from photon conversions or
decays of long lived particles in the Inner Detector, as these tracks will not
produce seeds. Additionally, inside out-tracking will sometimes fail due to
ambiguous hits shadowing the track seed in the densely populated inner layers of
the silicon detectors. Further, high energy loss at the outer radii of the SCT may cause
the track to change direction in the bending plane and the extension search to
go in the wrong direction. 

A complimentary tracking procedure called `outside-in'
tracking attempts to solve these problems by starting from the TRT and working
inwards. It begins by searching for track segments in the TRT using hits not
associated with a silicon track extension, using a Hough transform to identify
tracks~\cite{Baines:683897}. These track segments are then fitted using a Kalman
filter to take into account the drift-time measurements. These are then extended back into the
SCT and Pixel detectors.

\section{Vertex Finding}
\label{sec:reco-vertexing}

Location of interaction vertices is an important ingredient to precision
particle-physics measurements. It is important to know which particles were
associated with the primary interaction vertex, and parameters such as the
longitudinal and transverse impact parameters can be used to distinguish signal
leptons from fakes, leptons from conversions or secondary decays in jets.

In the ATLAS reconstruction process, vertex finding occurs after construction of
Inner Detector tracks, as described in \sec{reco-tracking}. The
vertex finding algorithm must associate tracks with primary vertices, and obtain
a best fit for the vertex position and its uncertainty. Two approaches are
used in ATLAS for associating tracks to vertices~\cite{1742-6596-119-3-032033}. 

In `finding-after-fitting',
tracks are preselected by requiring that they are consistent with the collision
region. The tracks are ordered by their longitudinal impact parameter, and a
sliding window algorithm is used to identify clusters of tracks. A fit is 
carried out to obtain the vertex position. Outlier tracks are then removed if
they have a \chisquared\ with a probability of being consistent with the vertex
of less than 8\%. The vertex is the refitted, and the process iterates until
there are no outliers left, or the cluster becomes too small. In this procedure
the maximum number of vertices produced is determined at the clustering stage, and
once discarded, outlier tracks are not used again. This can be sub-optimal in
busy environments if the initial seeding does not correctly identify clusters.

An alternative approach, with better handling of outliers, is
`finding-through-fitting'. This is the default approach in ATLAS. Tracks are
again preselected by consistency with the interaction region, but in this case a
single seed vertex is formed out of all the preselected tracks. This is fitted, and
tracks identified as outliers in the fit are used to create a second vertex
seed. A simultaneous fit is then carried out of the two vertices, and again
outlier tracks are used to create a new primary vertex. The procedure is
iterated until none of the remaining outliers fit with any vertex with a
\chisquared\ probability of more than 1\%.

An `Adaptive Vertex Finding'~\cite{0954-3899-34-12-N01} algorithm is used in
both cases for the vertex position fitting. 
This uses a Kalman filter to minimise the least squares distances of the tracks from
the vertex position. After a preliminary fit, tracks are assigned a
weight depending on their compatibility with the vertex, with outlier tracks
being down-weighted so as to have less of a pull on the vertex position. The
process is iterated until convergence.

\section{Cluster Reconstruction}
\label{sec:reco-clustering}

\section{Electron Reconstruction and Identification}
\label{sec:reco-el}

\subsection{Electron triggers}
\label{sec:reco-el-triggers}

\section{Muon Reconstruction and Identification}
\label{sec:reco-mu}

\subsection{Muon triggers}
\label{sec:reco-mu-triggers}
