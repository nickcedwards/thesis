Location of interaction vertices is an important ingredient to precision
particle-physics measurements. It is important to know which particles were
associated with the primary interaction vertex, and parameters such as the
longitudinal and transverse impact parameters can be used to distinguish signal
leptons from fakes, leptons from conversions, or secondary decays in jets.

In the ATLAS reconstruction process, vertex-finding occurs after reconstruction of
\id\ tracks, as described in \sec{reco-tracking}. The
vertex-finding algorithm must associate tracks with primary vertices, and obtain
a best fit for the vertex position and its uncertainty. Two approaches are
used in ATLAS for associating tracks with vertices~\cite{1742-6596-119-3-032033}. 

In `finding-after-fitting',
tracks are preselected by requiring that they are consistent with the collision
region. The tracks are ordered by their longitudinal impact parameter, and a
sliding window algorithm is used to identify clusters of tracks. A fit is 
carried out to obtain the vertex position. Outlier tracks are then removed if
they have a \chisquared\ with a probability of being consistent with the vertex
of less than 8\%. The vertex is then refitted, and the process iterates until
there are no outliers left, or the cluster becomes too small. In this procedure
the maximum number of vertices produced is determined at the clustering stage, and
once discarded, outlier tracks are not used again. This can be sub-optimal in
busy environments if the initial seeding does not correctly identify clusters.

An alternative approach, with better handling of outliers, is
`finding-through-fitting'. This is the default approach in ATLAS. Tracks are
again preselected by consistency with the interaction region, but in this case a
single seed vertex is formed out of all the preselected tracks. This is fitted, and
tracks identified as outliers in the fit are used to create a second vertex
seed. A simultaneous fit is then carried out of the two vertices, and again
outlier tracks are used to create a new primary vertex. The procedure is
iterated until none of the remaining outliers fits with any vertex with a
\chisquared\ probability of more than 1\%.

An `Adaptive Vertex Finding'~\cite{0954-3899-34-12-N01} algorithm is used in
both cases for the vertex position fitting. 
This uses a Kalman filter to minimise the least squares distances of the tracks from
the vertex position. After a preliminary fit, tracks are assigned a
weight depending on their compatibility with the vertex, with outlier tracks
being down-weighted so as to have less of a pull on the vertex position. The
process is iterated until convergence.
