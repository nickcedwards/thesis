Partons produced in particle interactions are not physically observable since
they hadronize and produce a collimated shower of particles known as a jet. In
order to reconstruct the jet in a physically meaningful way, it is necessary to
specify an algorithm to associate multiple energy deposits in the calorimeters
to a single jet (\intro{clustering}) and a recombination scheme for how to combine their
four-momentum. The ATLAS hadronic calorimeters are non-compensating, and do not
compensate from energy lost from the hadronic shower due to production of
secondary particles such as neutrinos and muons that escape from the calorimeter
and due to leakage of the shower out of the calorimeter. It is thus necessary to
calibrate the response of the calorimeter to a hadronic shower.

Jet algorithms need to be theoretically well behaved with respect to QCD
divergences. In particular, it is important that the jets produced are not
effected by soft emissions (the algorithm must be \intro{collinear safe}) and
are not affected if a parton splits into two collinear partons (the algorithm
must be \intro{collinear safe}). Additionally, the algorithm must give the same
physics results regardless of whether the input is partons, particles (from
Monte-Carlo simulation) or calorimeter objects. There are two main classes of
jet clustering algorithm: cone algorithms and successive combination algorithms.
Cone algorithms start from seed clusters and add in all other clusters within a
cone of a specified size in $\delta R$, where $\delta R = \sqrt{(\Delta \eta)^{2} + (\Delta
\phi)^{2}} $. Cone algorithms are generally theoretically unsafe as soft or
collinear emissions can affect the choice of seeds. Successive combination
iteratively merge pairs of objects according to a definition of distance that
generally involves the distance between the objects in and their transverse
momentum.

In ATLAS, the default jet clustering algorithm is a successive combination
algorithm called \antikt~\cite{1126-6708-2008-04-063}. This combines objects
according to the distance parameters $d_{i,j} =
\rm{min}(p_{T,i}^{-2},p_{T,j}^{-2}) \frac{ \Delta R }{R}$ and $d_{i,\rm{beam}} =
p_{T,i}^{-2}$ where $p_{T,i}$ is the transverse momentum of object $i$ and
$\delta R$ is the distance between the objects in $\eta, \phi$ as defined above.
The parameter $R$ is a parameter controlling the size of the jet, and is
analageous to the cone size in a cone based jet algorithm. The algorithm starts
with a list of all objects, and calculates $d_{i,j}$ for all pairs of objects and
$d_{i,{\rm beam}}$ for all objects. If the minimum of all $\{ d_{i,j}, d_{i,{\rm
beam}} \}$ is a  $d_{i,j}$ the objects will be merged; if it is a $ d_{i,{\rm
beam}}$ the object will be considered a jet and removed from the list. This
process is repeated until there are no object remaining. This algorithm is
collinear and infrared safe as soft radiation is clustered to harder objects
first. It results in regular conical jets which is experimentally desirable
since it gives a well defined jet area that can be used to pileup supression.
Where jets are used in this thesis, the $R$ parameter is set to be 0.4.

ATLAS uses anti-kt:
	- This jet algorithm is based on the comparison for all possible pairs of objects between
the distances di,j = min(p−2,p−2)∆R and di,Beam = p−2, where pT is the transverse momentum T,i T,j R T,i
of the particle considered, ∆R is the spatial distance of the two objects in η − φ space and R is a parameter of the algorithm which can be considered as a weight to the spatial distance of the two objects - the larger R, the wider the jets. In the following, R will be referred to as the jet algorithm distance parameter. If the minimum between all di,j and di,Beam is a di,j, objects i and j will be recombined, otherwise object i is considered a jet and not considered in the next iteration. With this algorithm, soft radiation is effectively clustered around the harder core of the process. Anti-kt jets have a regular, conical shape, experimentally desirable since it allows for a well defined jet area that can be used for e.g. pile-up subtraction.
	- theoretically safe
	- better performance before and after calibration (e.g. trigger matchin)
	- ATLAS uses 0.4 and 0.6
	-  final four-momentum of the jet is obtained from summing the four-momenta of its constituents in the four-vector recombination scheme. This scheme conserves energy and momentum, and allows a meaningful definition for the jet mass.


