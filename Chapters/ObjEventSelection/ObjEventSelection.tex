\graphicspath{{Chapters/ObjEventSelection/Figures/}}
\chapter{Object and Event Selection}
\label{chap:ObjEventSelection}

\section{Electron Selection}
\label{sec:objsel-el}

Electron reconstruction and identification was described in~\sec{reco-el}. Both
central electrons with \modetalt{2.47} and forward electrons are used. In
addition to the indentification requirements described in~\sec{reco-el}
additional selection requirements are imposed to select electrons likely to have
originated from \Z\ boson decays and to reject backgrounds. The requirements are
slightly different for the 8 \tev\ analysis compared to the 7 \tev\ analysis,
reflecting the different experimental conditions (such as higher pile-up in
2012) as well as optimisations made in 2012. The electron selection requirements
are summarised in~\tab{objsel-el} are described in more detail below. 

\begin{table}[!htbp]
  \centering
%   \vspace*{-1cm}
%\small
  \begin{tabular}{ l  l l }
    \hline\hline 
      Requirement        & 7 \tev\ & 8 \tev\ \\ 
      \hline
      \bf{Central Electron Selection} & \\
      Algorthim             & Standard (with GSF refit)     & Standard \\
      Quality               & Good Data Quality & \it{Same} \\
      ID cut                & \loosePP & \it{Same}       \\
      $\eta$                & $|\eta|<2.47$ & \it{Same} \\
      $E_T$                 & $E_T > 7$ GeV & \it{Same} \\
      $z_0$                 & $z_0 < 2$ mm & \it{Same} \\
      $d_0$                 & $|d_0|/\sigma(d_0) < 6 $ & \it{Same} \\
      Track isolation       & \ptconetwentylt{0.15} & \it{Same}   \\
      Calorimeter isolation & \etconetwentylt{0.3}          & \it{Not Applied} \\
      Overlap removal       & \multicolumn{1}{p{6cm}}{a) Remove $e$ if $\Delta R < 0.1$ from $\mu$} & \it{Same} \\
                            & \multicolumn{1}{p{6cm}}{b) Remove lowest $E_T$ $e$ in \deltaRlt{0.1} from another $e$} & \it{Same} \\ 
      \hline
      \bf{Forward Electron Selection:} & \\
      Alogirthm             & Forward & \it{Same} \\
      Quality               & Good Data Quality & \it{Same}  \\
      ID cut                & Tight & \it{Same} \\
      $\eta$                & $2.50<|\eta|<3.16$ & \it{Same} \\
      $E_T$                 & $E_T > 20$ GeV & \it{Same} \\
      Overlap removal       & \multicolumn{1}{p{6cm}}{Remove if overlaps with central electron or any muon in \deltaRlt{0.1}} & \it{Same}\\
%
      %\bf{Central Electron Selection} & \\
      %Algorthim             & Standard (with GSF refit)     & Standard \\
      %Quality               & \multicolumn{2}{c}{\texttt{(OQ  AND 1446 == 0})} \\
      %ID cut                & \multicolumn{2}{c}{\loosePP}       \\
      %$\eta$                & \multicolumn{2}{c}{$|\eta|<2.47$} \\
      %$E_T$                 & \multicolumn{2}{c}{$E_T > 7$ GeV} \\
      %$z_0$                 & \multicolumn{2}{c}{$z_0 < 2$ mm} \\
      %$d_0$                 & \multicolumn{2}{c}{$|d_0|/\sigma(d_0) < 6 $} \\
      %Track isolation       & \multicolumn{2}{c}{\ptconetwentylt{0.15}}   \\
      %Calorimeter isolation & \etconetwentylt{0.3}          & \it{Not Applied} \\
      %Overlap removal       & \multicolumn{2}{c}{a) Remove $e$ if $\Delta R < 0.1$ from $\mu$} \\
      %                      & \multicolumn{2}{c}{b) Remove lowest $E_T$ $e$ in \deltaRlt{0.1} from another $e$} \\ 
      %\hline
      %\bf{Forward Electron Selection:} & \\
      %Alogirthm             & \multicolumn{2}{c}{Forward} \\
      %Quality               & \multicolumn{2}{c}{\texttt{(OQ  AND 1446 == 0)}}  \\
      %ID cut                & \multicolumn{2}{c}{Tight} \\
      %$\eta$                & \multicolumn{2}{c}{$2.50<|\eta|<3.16$} \\
      %$E_T$                 & \multicolumn{2}{c}{$E_T > 20$ GeV} \\
      %Overlap removal       & \multicolumn{2}{p{8cm}}{\centering Remove if overlaps with central electron or any muon in \deltaRlt{0.1}} \\
    \hline \hline
  \end{tabular}
   \caption{Electron selection requirements.}
   \label{table:objsel-el}
\end{table}

\section{Central Electron Selection}

`Central' electrons, with \modetaclusterlt{2.47}, are reconstructed using the
``standard'' electron algorithm as described in~\sec{reco-el-reco}.  For 2011
data, the algorithm used was slightly different to the `standard' ATLAS electron
econstruction as tracks were refitted using a Gaussian-sum filter (GSF) to
account to account for the effect of bremsstrahlung in the inner detector. In
2012 data this became the default reconstruction algorithm. Central electrons
are required to pass the \loosePP\ identification algorithm.
%Electrons in the calorimeter ``crack" region $1.37 < |\eta_{cluster}| < 1.52$
%are included, although the efficiency and energy resolution is expected

For electron candidates with 4 or more silicon (SCT and Pixel) hits, the energy
of the electron is taken from the cluster measurement, and the eta and phi are
taken from the track (this requirement is automatically satisfied if the
electrons pass the \loosePP\ identification requirements). For electron
candidates with fewer than 4 silicon hits, all electron parameters are taken
directly from the cluster. In both cases, the cluster eta and phi are used for
the $\eta$ requirement and for overlap removal. Using the energy and direction
defined in this way, the electron candidates are required to have \etgt{7},
where \et\ is defined as $E\sin{\theta}=E/\cosh{\eta}$. 

A small number of the front-end boards of the liquid argon calorimeter were
inactive in 2011 and 2012; the exact number varied with time as some were
repaired whilst other developed faults. Additional, a number of individual cells
are masked in the readout (i.e. have their energy set to zero) due to
consistently producing high noise or failing to give a readout. There are also
regions where the High-Voltage supply to the LAr is faulty. Electrons are
rejected if they fall into a region of
$\eta, \phi$ space consistent with the presence of a dead front-end board in
the first or second sampling layer, the presence of a dead HV region affecting the
three samplings, or the presence of a masked cell in the core of the cluster.

To ensure that the candidates come from the primary vertex (defined as the
vertex that has the highest $\sum{\pt^2}$ of associated tracks), the
longitudinal impact parameter \zzero\ of the electron track with respect to the
primary vertex must be less than 2 mm. The \intro{unbiased} impact paramter is
used; the unbiased impact parameter for a particular track is obtained by
refitting the vertex without the track in question, then calculating the impact
parameter with respect to this refitted vertex, thus removing the pull of the
track from the vertex fit.  The transverse impact parameter \dzero\ must have a
significance (\dzero\ divided by the error on its measurement, \dzerosig) less
than 6.

%, see Sec.~\ref{sec:EventSelection}.
%the electron which is trigger matched
%to an e20\_medium trigger 

Similarly to our requirements on muons, electrons are required to be isolated both in the tracker and in the calorimeter.
Electron tracks must satisfy $ptCone30/E_{T}<0.15$ for the \llvv\ final state
and $ptCone20/E_{T}<0.15$ for the \llll\ final statet. 

In addition, calorimeter isolation cuts are applied. In the \llll\ final state, the ratio of the sum of the transverse energy in calorimeter cells 
within a cone of $\Delta R < 0.2$ around the electron 
 must be less than 30\% of the electron $E_T$ .
In the \llvv\ final state, the requirement is that the sum of the calorimeter transverse energy in
a cone of $\Delta R = 0.3$ around the electron candidate must be less than
15\% of the electron $E_{T}$.
In both cases, the energy in the isolation cone does not include the energy of
the electron itself, and is corrected for the
effects of pileup using the official prescription of the 
egamma performance group\footnote{We use $el\_Etcone20\_corrected = CaloIsoCorrection::GetPtNPVCorrectedIsolation(\ldots)$ from the egammaAnalysisUtils-00-03-24 package.}.

Electrons closer than \deltaRlt{0.1} to a muon which passes the object selection
requirements (see~\sec{objsel-mu}) are rejected. If two selected electrons
overlap within \deltaRlt{0.1}, the lower-\et\ electron is removed, although in
the case of a central electron overlapping with a forward electron, which could
occur near the edge of the tracker, the central electron would take precedence.

\section{Forward Electron Selection}

Electrons reconstructed using the forward electron reconstruction algorithm are
used to to extend the pesudo-rapidity coverage beyond the limit of the tracker,
\modetaeq{2.5}. Only elecrons falling in the EMEC region are used, with
pseudo-rapidity \modetaclusterbetween{2.5}{3.16}.  Since the lack of tracking makes it
harder to reject hadronic and photonic fakes, these electrons are required to
pass tighter identification requirements: ``Forward Tight''. Since these
electrons lack a track measurement, it is impossible to determine their charge,
and of course impossible to apply track isolation and track parameter cuts.
Forward electrons are required to have \etgt{20}; this higher energy requirement
is motivated by the difficulty of measuring reconstruction and identification
efficiencies in data below this energy and by the higher hadronic and photonic
backgrounds at low energy.
%For this reason a tighter calorimeter isolation requirement is applied: the sum
%of the calorimeter transverse energy in a cone of $\Delta R = 0.3$ around the
%electron, removing the electron energy, must be less than 10\% of the electron
%energy.

\section{Muon Selection}
\section{Jet Selection}
\section{\ZZ\ Event Selection}
\subsection{Triggers}
\section{Selection Efficiencies}
\subsection{\CZZ}
\subsection{Mispairing rates}
\subsection{Tau Contribution}
\section{Observed Kinematic Distributions}
