\graphicspath{{Chapters/BackgroundEstimate/Figures/}}
\chapter{Background Estimate}
\label{chap:BackgroundEstimate}

The four-lepton final state is expected to be very clean with little
background, since few other \sm\ interactions produce four high-\pt\ isolated leptons
in the final state. Almost all of the sources of background include one or more
\intro{background leptons}, where a background
lepton is defined as either a fake-lepton reconstructed due to jets or
photons mis-identified as a lepton, or a real lepton from decays within jets or
from photon conversions.
The dominant background contribution is expected to arise from the production of a \Z\ in
association with jets and or photons (termed \Zjets\ and \Zgamma\ below). Other
contributions arise from top-quark production (\ttbar and \singletop) and from
other diboson processes \WW+jets and \WZ+jets.
%\ttbar$\ra\W\W bb$, $\Wt\ra\W\W b$, $\Zt\ra\W\Z b$, \WW\ and \WZ.

A very small but background arises from $ZZZ$ and $WZZ$ production
and from \ttbar+$V$ where $V=\W,\Z$. In these backgrounds there are four leptons
from \W\ or \Z\ boson decays, so they will tend to be isolated and have small
impact parameters, making these irreducible sources of background.
The \cx\ for these processes are, however, very small, so they contribute only a
small fraction of the overall background. 
%\CX s for the background sources described above are given in~\tab{}.

\mcsim\ can be used to estimate the size of the background, however this relies
on accurate modelling of particle production within jets. Accurate modelling is
required so that the rate of
lepton production from hadronic decays within jets is modelled correctly, and
the shower shapes of jets in the calorimeters are well modelled so that the
background due to jets faking the electron identificaton is well modelled. In
order for leptons or fakes to pass the selection requirements they must pass the
isolation requirement. This means background leptons will tend to be in the
tails
of the jet distribution. The \mc\ is not expected to
perform well in this area, as it relies heavily on the model used for
hadronisation and on details of the parton shower. A data-driven technique is
therefore used to estimate the background from events with one or two background
leptons. This estimate estimates the expected combined background from \Zjets,
\Zgamma\, \WW, \WZ, \ttbar\ and \singletop. \mc\ is used to estimate the
irreducible background, and as a cross check to the data-driven estimate.

\mc\ based background estimates are
described in~\sec{mcbg}, and the data-driven estimate is described in~\sec{ddbg}.

\section{\mc\ Background Estimates}
\label{sec:mcbg}

The \mc\ generators used to model the different sources of background are listed
in~\tab{mcbg-generators}. In a few cases different generators were used for the
8~\tev\ analysis with respect to the 7~\tev\ analysis, owing to developments in
the avilable \mc\ generators. \Zjets\ samples generated with \alpgen\ are
normalised to the inclusive NNLO \cx\ prediction of the FEWZ
program~\cite{Gavin:2010az}. The \ttbar\ samples are normalised to the
approximate NNLO calculation of HATHOR~\cite{Aliev:2010zk}. Other samples are
normalised to the \cx\ predictions of the generator used to produce them.
%The majority of the generators calculate the \cx\
%to NLO

\begin{table}
\centering
\small
  \begin{tabular}{lcccc}
    \hline\hline
     Process & Generator 7~\tev\ & Generator 8~\tev \\
     \hline
     \Zjets & \alpgen+\jimmy           & \alpgen+\jimmy \\
     \ttbar & \mcatnlo+\jimmy           & \mcatnlo+\jimmy \\
     \singletop & \acermc+\jimmy           & \acermc+\pythia \\
     \WZ        & \mcatnlo+\jimmy & \powhegbox+\pythia \\
     \WW        & \mcatnlo+\jimmy, \ggtwoWW & \powhegbox+\pythia, \ggtwoWW \\
     $ZZZ/WWZ$  & Not Used      & \madgraph+\pythia \\
     \ttbar+$V$     & Not Used  & \madgraph+\pythia \\
    \hline\hline
  \end{tabular}

      \caption[\mc\ generators used to model background processes.]
      {\mc\ generators used to model background processes. }
\label{table:mcbg-generators}
\end{table}

The \mc\ estimated background for the 7~\tev\ analysis is shown
in Tables~\ref{table:mc-bg-4e},~\ref{table:mc-bg-4mu} and
~\ref{table:mc-bg-2e2mu} for the \eeee, \mmmm\ and \eemm\ final-states,
respectively, and in~\tab{mc-bg-4l} for all final-states together. The
background estimates are all statistically limits, with typically only one or
two events passing all of the selections. In the \eeee\ final-state the
background is seen to mainly arise from \Zjets, with a smaller contribution from
\WZ\ and \WW. The background to the \ZZs\ selection is significantly larger than
the background to the \ZZ\ selection, as the tighter mass cut applied in the
\ZZ\ selection rejects backgrounds where the second \Z\ candidate is formed from
background leptons. The background to the \mmmm\ final-state is seen to be
predicted to be much smaller, with the only contribution arising in the \mc\
from \WZ\ events. The total esimated background to the \ZZs\ selection is
\measStat{8.3}{\errSym{1.3}}, and the estimated background to the \ZZ\ selection
is \measStat{1.5}{\errSym{0.4}}. Since this estimated is only used as a
cross-check to the data-drive estimate, and due to lack of statistics,
systematic uncertainties on these background estimates are not evaluated.

%%% 4e
\begin{table}[htbp]
  \centering
  \begin{tabular}{r|c|c|c} 
    \hline\hline
                 Cut &               $Z$+jets &             $WZ/WW$ &               Top\\ 
    \hline
        Four Leptons        &  12.2 $\pm$ 1.8 & 0.8 $\pm$ 0.4 & 0.2 $\pm$ 0.2 \\ 
       Trigger Match        &  11.2 $\pm$ 1.8 & 0.8 $\pm$ 0.4 & 0.2 $\pm$ 0.2 \\ 
       2 OS-SF Pairs        &  7.0  $\pm$ 1.5 & 0.6 $\pm$ 0.2 & 0.2 $\pm$ 0.2 \\ 
66 $ < M_{Z1} < $ 116 GeV   &  5.1  $\pm$ 1.2 & 0.5 $\pm$ 0.2 & $<0.2$ \\ 
  $M_{Z2} > $ 20 GeV        &  3.5  $\pm$ 0.9 & 0.3 $\pm$ 0.1 & $<0.2$ \\ 
66 $ < M_{Z2} < $ 116 GeV   &  0.6  $\pm$ 0.2 & 0.1 $\pm$ 0.1 & $<0.2$ \\ 
    \hline\hline
  \end{tabular}
  \caption[MC predicted number of events passing various levels of selection for
  the \Zjets, \WZ/\WW and \topquark\ backgrounds in the \eeee\ final-state.]
  {MC predicted number of events passing various levels of selection for
  the \Zjets, \WZ/\WW and \topquark\ backgrounds in the \eeee\ final-state. The
  \Zjets\ background includes contributions from both light and heavy flavour
  jets. The top quark background includes contributions from \ttbar\ and
  single top. The yields are normalised to 4.7~\ifb.
  }
  \label{table:mc-bg-4e}
\end{table}

%%% 4mu
\begin{table}[htbp]
  \centering
  \begin{tabular}{r|c|c|c} 
    \hline\hline
                 Cut &               $Z$+jets &             $WZ/WW$ &               Top\\ 
    \hline

        Four Leptons        &  0.3 $\pm$ 0.3 & 0.1 $\pm$ 0.1    & - \\ 
       Trigger Match        &  0.3 $\pm$ 0.3 & 0.1 $\pm$ 0.1    & - \\ 
       2 OS-SF Pairs        &  0.3 $\pm$ 0.3 & 0.1 $\pm$ 0.1    & - \\ 
66 $ < M_{Z1} < $ 116 GeV   &  < 0.3         & 0.1 $\pm$ 0.1    & - \\ 
  $M_{Z2} > $ 20 GeV        &  < 0.3         & 0.1 $\pm$ 0.1    & - \\ 
66 $ < M_{Z2} < $ 116 GeV   &  < 0.3         & $<0.1$           & - \\ 
    \hline\hline
  \end{tabular}
  \caption[MC predicted number of events passing various levels of selection for
  the \Zjets, \WZ/\WW and \topquark\ backgrounds in the  \mmmm\ final-state.]
  {MC predicted number of events passing various levels of selection for
  the \Zjets, \WZ/\WW and \topquark\ backgrounds in the \mmmm\ final-state. The
  \Zjets\ background includes contributions from both light and heavy flavour
  jets. The top quark background includes contributions from \ttbar\ and
  single top. The yields are normalised to 4.7~\ifb.
  }
  \label{table:mc-bg-4mu}
\end{table}

%%% 2e2mu
\begin{table}[htbp]
  \centering
  \begin{tabular}{r|c|c|c} 
    \hline\hline
                 Cut &               $Z$+jets &             $WZ/WW$ &               Top\\ 
    \hline
        Four Leptons        &  21.2 $\pm$ 2.9  & 1.2 $\pm$ 0.2 & 0.1 $\pm$ 0.1 \\ 
       Trigger Match        &  20.8 $\pm$ 2.8  & 1.2 $\pm$ 0.2 & 0.1 $\pm$ 0.1 \\ 
       2 OS-SF Pairs        &  7.0  $\pm$ 1.2  & 0.7 $\pm$ 0.2 & $<$ 0.1 \\ 
66 $ < M_{Z1} < $ 116 GeV   &  4.9  $\pm$ 1.0  & 0.6 $\pm$ 0.2 & $<$ 0.1 \\ 
  $M_{Z2} > $ 20 GeV        &  4.0  $\pm$ 0.9  & 0.5 $\pm$ 0.1 & $<$ 0.1 \\ 
66 $ < M_{Z2} < $ 116 GeV   &  0.7  $\pm$ 0.3  & 0.1 $\pm$ 0.1 & $<$ 0.1 \\ 
    \hline\hline
  \end{tabular}
  \caption[MC predicted number of events passing various levels of selection for
  the \Zjets, \WZ/\WW and \topquark\ backgrounds in the \eemm\ final-state.]
  {MC predicted number of events passing various levels of selection for
  the \Zjets, \WZ/\WW and \topquark\ backgrounds in the \eemm\ final-state. The
  \Zjets\ background includes contributions from both light and heavy flavour
  jets. The top quark background includes contributions from \ttbar\ and
  single top. The yields are normalised to 4.7~\ifb.
  }
  \label{table:mc-bg-2e2mu}
\end{table}

%%% 4l
\begin{table}[htbp]
  \centering
  \begin{tabular}{r|c|c|c} 
    \hline\hline
                 Cut &               $Z$+jets &             $WZ/WW$ &               Top\\ 
    \hline
        Four Leptons        &  34.5 $\pm$ 3.4 & 2.0 $\pm$ 0.4 & 0.3 $\pm$ 0.2 \\ 
       Trigger Match        &  33.0 $\pm$ 3.3 & 2.0 $\pm$ 0.4 & 0.3 $\pm$ 0.2 \\ 
       2 OS-SF Pairs        &  14.3 $\pm$ 1.9 & 1.3 $\pm$ 0.3 & 0.2 $\pm$ 0.2 \\ 
66 $ < M_{Z1} < $ 116 GeV   &  10.0 $\pm$ 1.6 & 1.2 $\pm$ 0.3 & $<$ 0.2 \\ 
  $M_{Z2} > $ 20 GeV        &  7.4  $\pm$ 1.3 & 0.8 $\pm$ 0.2 & $<$ 0.2 \\ 
66 $ < M_{Z2} < $ 116 GeV   &  1.2  $\pm$ 0.4 & 0.3 $\pm$ 0.1 & $<$ 0.2 \\ 
    \hline\hline
  \end{tabular}
  \caption[MC predicted number of events passing various levels of selection for
  the \Zjets, \WZ/\WW and \topquark\ backgrounds in all \llll\ final-states
  combined.]
  {MC predicted number of events passing various levels of selection for
  the \Zjets, \WZ/\WW and \topquark\ backgrounds in all \llll\ final-states
  combibned. The
  \Zjets\ background includes contributions from both light and heavy flavour
  jets. The top quark background includes contributions from \ttbar\ and
  single top. The yields are normalised to 4.7~\ifb.
  }
  \label{table:mc-bg-4l}
\end{table}

\section{Data Driven Background Estimates}
\label{sec:ddbg}

\subsection{Methodology}

As described above, the reducible background sources fail into two categories:

\begin{itemize}
\item Backgrounds with two prompt isolated leptons and two `background' leptons. Such
background include \Zjets, \Zgamma, \WW+jets, \ttbar\ and \singletop\ (in the
$s$ and $t$ channels).
\item Backgrounds with three prompt isolated leptons and one `background'
lepton. Such backgrounds include \WZ+jets and \singletop production in the \Wt\
channel.
\end{itemize}

Denoting true leptons as $T$ and background leptons as $B$, the total background due
to fake (background) leptons can be experessed as:

\begin{equation}
N_{4l}^{\rm fake} = N_{TTTB} \times f + N_{TTBB} \times f^{2}
\label{eqn:bg-4l-true}
\end{equation}

where $f$ is the \frate, the fraction of background leptons that pass the lepton selection
requirements. Of course, given a selected lepton in data it is impossible to
know whether it is a true lepton or a background lepton (if it were, background
rejection would be trivial). Instead, in order to measure the background 
two new definitions are introduced: \intro{selected leptons}, denoted $L$, that
pass all of the lepton selection requirements; and \intro{lepton-like-jets} $J$, which
pass most of the selection requirements, but fail a few selected requirements.
The background is estimated by extrapolating from control regions containing
two or three \sellep\ s and one one or two \lljet (s) using the
\intro{\ffactor}\ \FF, defined as the ratio of the probability for a background lepton to be
classified as a \sellep\ to the probability for it to be classified as a \lljet.
In a sample containing only \bglep s, the \ffactor\ and \frate s are given by: 

\begin{equation}
f = \frac{N_{L}}{N_{L} + N_{J}},\ \FF = \frac{N_{L}}{N_{J}}
\end{equation}
where $N_{L}$ and $N_{J}$ are the number of \sellep s and \lljet s in the sample,
respectively. \FF\ and \f\ are thus related by:

\begin{equation}
\f = \frac{\FF}{1 + \FF},\ \FF = \frac{\f}{1-\f}
\label{eqn:bg-f-FF-relations}
\end{equation}
%The number of events with two \lljet s and two \sellep\ s is related to the
%number of events with two true leptons and two \bglep s by:
The number of events with two $L$ and two $J$ is related to the
number of events with two $T$ and two $B$ by:

\begin{equation}
N_{LLJJ} = N_{TTBB} \times (1-f)^{2}
\end{equation}
%Similary, the number of events with three \sellep s and one \lljet\ can be related to the
%number of events with two or three true leptons and two or three \bglep s by:

Similary, the number of events with three $L$ and one $J$ can be related to the
number of events with two or three $T$ and one or two $B$ \bglep s by:

\begin{equation}
N_{LLLJ} = N_{TTTB} \times (1-f) + N_{TTBB} \times 2f(1-f)
\end{equation}
The background to the four-lepton selection given in~\eqn{bg-4l-true} can thus
be rewritten in terms of the number of events in the $LLLJ$ and $LLJJ$ control
regions:
\begin{align}
N_{4l}^{\rm fake} &= N_{LLLJ} \times \FF - N_{LLJJ} \times \FF^{2} \\
&= N_{LLLJ} \times \frac{\f}{(1-\f)} - N_{LLJJ} \times \frac{\f^{2}}{(1-\f)^{2}}
&=  N_{TTTB} \times \frac{\f^{2}}{(1-\f)} + N_{TTBB}
\times 2 f^{3} / (1-f) - N_{TTBB} \times \frac{\f} -
N_{TTTB} \times f + N_{TTBB} \times f^{2} \\
&= N_{}
\end{align}

\subsection{Fake-Factor Measurement}
\subsection{Control Plots}
\subsection{Results}
