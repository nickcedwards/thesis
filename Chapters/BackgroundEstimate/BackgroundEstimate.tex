\graphicspath{{Chapters/BackgroundEstimate/Figures/}}
\chapter{Background Estimate}
\label{chap:BackgroundEstimate}

The four-lepton final state is expected to be very clean with little
background, since few other interactions produce four high-\pt\ isolated leptons
in the final state. Almost all of the sources of background include one or more
\intro{background leptons}, where a background
leptons is defined as either a fake-lepton reconstructed due to jets or
photons mis-identified as a lepton, or a real lepton from decays within jets or
from photon conversions.
The dominant background contribution is expected to arise from the production of a \Z\ in
association with jets and or photons (termed \Zjets\ and \Zgamma\ below). Other
contributions arise from top-quark production (\ttbar and \singletop) and from
other diboson processes \WW\ and \WZ.
%\ttbar$\ra\W\W bb$, $\Wt\ra\W\W b$, $\Zt\ra\W\Z b$, \WW\ and \WZ.

A very small but irreducible background arises from $ZZZ$ and $WZZ$ production.
Both of these processes produce large numbers of high-\pt\ isolated leptons,
however their \cx\ are very small so contribute a negigable amount to the
overall selection.

\CX s for the background sources described above are given in~\tab{}.

\mcsim\ can be used to estimate the size of the background, however this relies
on accurate modelling of particle production within jets, so that the rate of
lepton production from hadronic decays within jets is modelled correctly, 
that the shower shapes of jets in the calorimeters are well modelled so that the
background due to jets faking leptons is well modelled, and that the isolation
variables for leptons in jets are well modelled. The \mc\ is not expected to
perform well in this area, as it relies heavily on the model used for
hadronisation and on details of the parton shower. A data-driven technique is
therefore used to estimate the background. \mc\ based background estimates are
described in~\sec{mcbg}, and the data-driven estimate is described in~\sec{ddbg}.

\section{\mc\ Background Estimates}
\label{sec:mcbg}

The \mc\ generators used to model the different sources of background are listed
in~\tab{mcbg-generators}. In many cases different generators were used for the
8~\tev\ analysis with respect to the 7~\tev\ analysis.

\begin{table}
\centering
\small
  \begin{tabular}{lcccc}
    \hline\hline
     Process & Generator 7~\tev\ & Generator 8~\tev \\
     \hline
     \Zjets & \alpgen+\jimmy           & \alpgen+\jimmy \\
     \ttbar & \mcatnlo+\jimmy           & \mcatnlo+\jimmy \\
     \singletop & \mcatnlo+\jimmy           & \acermc+\pythia \\
     \WZ        & \mcatnlo+\jimmy
     %\Wt     & \acermc+\jimmy           & \acermc+\jimmy \\
    \hline\hline
  \end{tabular}

      \caption[\mc\ generators used to model background processes.]
      {\mc\ generators used to model background processes. }
\label{table:mcbg-generators}
\end{table}

\section{Data Driven Background Estimates}
\label{sec:ddbg}

\subsection{Methodology}
\subsection{Fake-Factor Measurement}
\subsection{Control Plots}
\subsection{Results}
