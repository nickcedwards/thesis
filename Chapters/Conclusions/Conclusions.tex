\graphicspath{{Chapters/Conclusions/Figures/}}
\chapter{Conclusions}
\label{chap:Conclusions}
The datasets collected by the ATLAS experiment in 2011 and 2012 have allowed the
measurement of the \ZZ\ production \cx s at \sqrtseq{7} and \sqrtseq{8}.

Events were selected which are consistent with two \Z\ bosons decaying to electrons or muons. The \cx\ is
first measured in a fiducial \phasespace\ corresponding closely to the detector
acceptance, in order to reduce theoretical uncertainties arising from
extrapolating to regions where there is no experimental acceptance.
%For the 7~\tev\ measurement, this volume is defined by requring four
%electrons or muons with \ptgt{7} and \modetalt{3.16}, with a minumum separation
%between any pair of leptons (electrons or muons) of \deltaRlt{0.2}. The leptons
%must form two \ossf\ pairs, each with invariant mass \sstooos. 
The fiducial \cx,  requring both \leppair s to have \sstooos, measured in a
dataset corresponding to an integrated  luminosity of
\LumiPassGRLTwentyEleven~\ifb\ is:
\begin{align}
\sigmaFidZZllll(\sqrtseq{7} &= \ZZSevenTeVFiducialCrossSectionZZLLLL \\
\intertext{to be compared with a theoretical expectation of:}
\sigmaFidTheoryZZllll(\sqrtseq{7} &= \ZZSevenTeVTheoryFiducialCrossSectionZZLLLL
\end{align}
Addionally, a fiducial \cx\ relaxing the mass requirement on one of the lepton pairs
to \mllgtt is measured. This is found to be:
\begin{align}
\sigmaFidZZsllll(\sqrtseq{7} &= \ZZSevenTeVFiducialCrossSectionZZLLLL \\
\intertext{to be compared with a theoretical expectation of:}
\sigmaFidZZsllll(\sqrtseq{7} &= \ZZSevenTeVTheoryFiducialCrossSectionZZLLLL
\end{align}
For the 8~\tev\ measurement, the fiducial volume is defined in the same way as
at 7~\tev\, except with the pseudo-rapidity requirement tightened to \modetalt{2.7}. The fiducial \cx\
measured in a dataset corresponding to a luminosity of
\LumiPassGRLTwentyTwelve~\ifb\ is:
\begin{align}
\sigmaFidZZllll &= \ZZEightTeVFiducialCrossSectionZZLLLL \\
\intertext{to be compared with a theoretical expectation of:}
\sigmaFidTheoryZZllll &= \ZZEightTeVTheoryFiducialCrossSectionZZLLLL
\end{align}

These results are then extrapolated to the total cross section for \ZZ\
production with \Z\ bosons in the mass range 66$\GeV$ to 116$\GeV$, by
correcting for the acceptance of the fiducial \phasespace\ and the \Zll\
branching fractions. The total \cx\ it \sqrtseq{7} is measured to be:
\begin{align}
\sigmaTotZZ(\sqrtseq{8}) &= \ZZSevenTeVTotalCrossSection \\
\intertext{to be compared with a theoretical expectation of:}
\sigmaTotTheoryZZ(\sqrtseq{8}) &= \ZZSevenTeVTheoryTotalCrossSection
\end{align}
At \sqrtseq{8} the total \cx\ is measured to be:
\begin{align}
\sigmaTotZZ(\sqrtseq{8}) &= \ZZEightTeVTotalCrossSection \\
\intertext{to be compared with a theoretical expectation of:}
\sigmaTotTheoryZZ(\sqrtseq{8}) &= \ZZEightTeVTheoryTotalCrossSection
\end{align}
Compare theory with expectation

Unfolded distributions

The differential event yield as a function of the transverse momentum of the
highest transeverse momentum \Z\ boson is used to set limits on the strength of
anomolous \ZZZ\ and \ZZg\ neutral triple gauge boson couplings, which are
forbidden in the \sm. 
%Limits are set separately with the \sqrtseq{7} and \sqrtseq{8} data. 
The limits obtained with the \sqrtseq{8} data are the most constraining to date.
