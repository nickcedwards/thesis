\graphicspath{{Chapters/Conclusions/Figures/}}
\chapter{Conclusions}
\label{chap:Conclusions}
Data collected by the ATLAS experiment in 2011 and 2012 were used to make
measurements of \ZZ\ production at \sqrtseq{7} and \sqrtseq{8}.
Events were selected which were consistent with two \Z\ bosons decaying to
electrons or muons. The \cx\ was measured in a fiducial \phasespace\
corresponding closely to the detector acceptance, in order to reduce theoretical
uncertainties arising from extrapolating to regions where there is no
experimental acceptance.
%For the 7~\tev\ measurement, this volume is defined by requring four
%electrons or muons with \ptgt{7} and \modetalt{3.16}, with a minumum separation
%between any pair of leptons (electrons or muons) of \deltaRlt{0.2}. The leptons
%must form two \ossf\ pairs, each with invariant mass \sstooos. 
The fiducial \phasespace\ for the \sqrtseq{7} measurement is defined by
requiring both \leppair s to have \sstooos, and requiring all leptons have \ptgt{7},
\modetalt{3.16} and be separated from any other lepton by \deltaRlt{0.2}. The
fiducial \cx\ measured in a
dataset corresponding to an integrated luminosity of
\LumiPassGRLTwentyEleven~\ifb was found to be:
\begin{align}
\sigmaFidZZllll(\sqrtseq{7}) &= \ZZSevenTeVFiducialCrossSectionZZLLLL \\
\intertext{to be compared with a theoretical expectation of:}
\sigmaFidTheoryZZllll(\sqrtseq{7}) &= \ZZSevenTeVTheoryFiducialCrossSectionZZLLLL
\end{align}
Additionally, a fiducial \cx\ relaxing the mass requirement on one of the lepton pairs
to \mllgtt was measured to be:
\begin{align}
\sigmaFidZZsllll(\sqrtseq{7}) &= \ZZSevenTeVFiducialCrossSectionZZsLLLL \\
\intertext{to be compared with a theoretical expectation of:}
\sigmaFidZZsllll(\sqrtseq{7}) &= \ZZSevenTeVTheoryFiducialCrossSectionZZsLLLL
\end{align}

At \sqrtseq{8}, the fiducial \phasespace\ is defined similarly to the 7~\tev\
measurement, but with the lepton pseudo-rapidity requirement tightened to
\modetalt{2.7}. The fiducial \cx\ measured in a dataset corresponding to an integrated luminosity of
\LumiPassGRLTwentyTwelve~\ifb\ was:
\begin{align}
\sigmaFidZZllll(\sqrtseq{8}) &= \ZZEightTeVFiducialCrossSectionZZLLLL \\
\intertext{to be compared with a theoretical expectation of:}
\sigmaFidTheoryZZllll(\sqrtseq{8}) &= \ZZEightTeVTheoryFiducialCrossSectionZZLLLL
\end{align}

The uncertainties on the measurements at both energies are statistically
dominated. The dominant systematic uncertainties arise due to uncertainties on
the lepton reconstruction and identification efficiencies. 

The fiducial \cx s were extrapolated to the total \cx\ for \ZZ\
production with \Z\ bosons in the mass range 66$\GeV$ to 116$\GeV$, 
correcting for the acceptance of the fiducial \phasespace\ and the \Zll\
branching fractions. The total \cx\ at \sqrtseq{7} was measured to be:
\begin{align}
\sigmaTotZZ(\sqrtseq{7}) &= \ZZSevenTeVTotalCrossSection \\
\intertext{to be compared with a theoretical expectation of:}
\sigmaTotTheoryZZ(\sqrtseq{7}) &= \ZZSevenTeVTheoryTotalCrossSection
\end{align}
At \sqrtseq{8} the total \cx\ was measured to be:
\begin{align}
\sigmaTotZZ(\sqrtseq{8}) &= \ZZEightTeVTotalCrossSection \\
\intertext{to be compared with a theoretical expectation of:}
\sigmaTotTheoryZZ(\sqrtseq{8}) &= \ZZEightTeVTheoryTotalCrossSection
\end{align}
The theoretical expectations are calculated to next-to-leading order in QCD, and
the error arises from uncertainties on the \partDF\ and the \fact\ and \renorm\
scales. The 7~\tev\ measurement is slightly higher than the theoretical prediction,
but consistent within the uncertainties. The 8~\tev\ measurement is in good
agreement with the theoretical prediction. 

The observed \ZZllll\ events were used to set limits on anomalous neutral triple
gauge couplings (\TGC s) by performing binned fits to kinematic distributions.
%The \ZZ\ production \cx\ would be enhanced by the presence of \ZZZ\ and \ZZg\
%vertices, which are forbidden in the \sm.
It was found that the differential event yield as a function of the transverse
momentum of the leading (in transverse momentum) \Z\ boson provided the
greatest sensitivity to \TGC s. The observed data were consistent with the non-existence of 
\ZZZ\ and \ZZg\ vertices, as predicted by the \sm, and so 95 \% confidence level
limits were set on the size of the \TGC\ couplings. The coupling are
described by four parameters: \ffourg, \ffourZ, \ffiveg\ and \ffiveZ. The
most stringent limits were obtained with the \sqrtseq{8} data, and were found to
be:
\begin{align}
-0.007 < \ffourg < 0.007, &\; -0.006 < \ffourZ < 0.006, \nonumber \\
-0.007 < \ffiveg < 0.007, &\; -0.006 < \ffiveZ < 0.006
\end{align}
assuming a \formfactor\ with cutoff scale $\Lambda=3 \tev$ and strength $n=3$.
These are the most constraining limits on \TGC s to date.
