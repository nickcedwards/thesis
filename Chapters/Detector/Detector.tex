\graphicspath{{Chapters/Detector/Figures/}}
\chapter{The ATLAS Detector at the LHC}
\label{chap:Detector}

ATLAS is a general purpose particle-physics detector at the Large Hadron
Collider (LHC)

\section{The Large Hadron Collider}

\section{The ATLAS Detector}

A cut-away view of the ATLAS detector is shown in Figure . The detector consists
of an inner tracking detector surrounding the interaction point, which is
surrounded by electromagnetic and hardonic calorimeters and finally a muon
spectrometer. A three-level trigger system is used to select events to read out.
These sub-systems are described in more detail in the following
sections. 


\begin{figure}[h]
\centering
\includegraphics[width=\textwidth]{{lhc-pho-1998-304}.jpg}
\caption{\deltat Cut-away view of the ATLAS detector~\cite{Jean-Luc:841458}. The
various detector sub-systems are labelled.}
\label{fig:atlas-det-1}
\end{figure}

\subsection{ATLAS Co-ordinate System}

\subsection{Inner Detector}

\subsubsection{Pixel Detector}
\subsubsection{Semiconductor Tracker (SCT)}
\subsubsection{Transition Radition Tracker (TRT)}
\subssection{Calorimetery}
\subssubsection{Electromagnetic Calorimeters}
\subssubsection{Hadronic Calorimeters}
\subssubsection{Muon Spectrometer (MS)}

\label{sec:Detector-SCT}

The Atlas Semi-Conductor Tracker (SCT) is one of three components of the experiment's Inner Detector (ID). The SCT is a silicon strip detector, designed to reconstruct the tracks of passing charged particles. The SCT consists of 4 barrels surrounding the beam pipe and two end-caps consisting of 9 disks each. The barrels consist of 2112 separate modules and extend from a radius of 299mm from the beam line at the innermost barrel to a radius of 514mm at the outermost. The barrels are numbered from 3 to 6. Each endcap consists of 988 modules arranged in such a way that a particle must pass through four layers of the detector.

The SCT modules are made from two pairs of single sided p-in-n silicon chips biased at 150V. Charged particles passing through the depletion region at the centre of the junction produce electron hole pairs. These are swept apart by the bias voltage. The electrons are then collected on the top of the chip, producing a signal which can be read out. Each pair of chips is wire-bonded together and two pairs are glued together to form a two sided module.

\section{Detector Simulation}
\section{Data Samples}
