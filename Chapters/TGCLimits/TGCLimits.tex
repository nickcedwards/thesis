\graphicspath{{Chapters/TGCLimits/Figures/}}
\chapter{Limits on Anomalous Neutral Triple Gauge Couplings}
\label{chap:TGCLimits}

\section{Limit Setting Procedure}

\section{Matrix Element Reweighting}
As described in~\sec{}, there are four anomoulous couplings for \ZZ\ production.
Since they enter linearly in the effective Lagrangian, they will appear
quadratically in the amplitude of the \ZZllll\ process. The differential \cx\
can thus be written as:
\begin{eqnarray}\label{eqn:dsigma}
{\rm d}\sigma_{\rm SM+TGC} & = & F_{00} + f_4^\gamma F_{01} + f_4^Z F_{02} + f_5^\gamma F_{03} + f_5^Z F_{04}  \nonumber \\
&+& \left(f_4^\gamma\right)^2F_{11} + f_4^\gamma f_4^Z F_{12} +  f_4^\gamma f_5^\gamma F_{13} + f_4^\gamma f_5^Z F_{14}  \nonumber \\
&+& \left(f_4^Z\right)^2F_{22} + f_4^Zf_5^\gamma F_{23} + f_4^Zf_5^Z F_{24}  \nonumber \\
&+& \left(f_5^\gamma\right)^2F_{33} + f_5^\gamma f_5^Z F_{34} \nonumber \\
&+& \left(f_5^Z\right)^2F_{44}
\end{eqnarray}
where $F_{ij}$ are coefficients. $F_{00}$ corresponds to the contribution of \sm
diagrams and the rest consist of operator contributions associated with the
anomalous couplings. Using this expression, an event generated assuming only
\sm\ couplings can be reweighted to the differential \cx\ assuming TGC couplings
at some specific value of the four couplings by applying the weight:
\begin{equation}
{\rm weight} = \frac{{\rm d}\sigma_{\rm SM+TGC}}{{\rm d}\sigma_{\rm SM}}
\end{equation}
 This procedure can be easily extended to reweight any sample generated assuming
a given set of TGCs to any other set of TGCs. For example, to reweight a sample
generated assuming only \sm\ couplings to sample assuming \ffourg=0.1, one would
apply to each event a weight:
\begin{eqnarray}\label{revisit}
{\rm weight} = \frac{F_{00} + 0.1 \cdot  F_{01} + 0.1\cdot 0.1 \cdot
F_{11}}{F_{00}}
\end{eqnarray}
The \Fij s are completely
specified by the kinematics of the incoming and outgoing particles, and so must
be evaluated on an event by event basis. They are
independent of the sizes of the anomalous couplings, although they do depend on
the choice of \formfactor.

How the Fij are evaluated

How to translate into yield coefficients

Since the TGCs enhance the \cx\

\section{Validation of Reweighting Procedure}

\section{Yield Coefficients}

\section{Bin Optimisation}

\section{Expected Limits}

\subsection{Observed Limits}
