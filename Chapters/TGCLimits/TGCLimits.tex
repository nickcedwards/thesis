\graphicspath{{Chapters/TGCLimits/Figures/}}
\chapter{Limits on Anomalous Neutral Triple Gauge Couplings}
\label{chap:TGCLimits}

\section{Limit Setting Procedure}

\subsection{Matrix Element Reweighting}
As described in~\sec{}, there are four anomoulous couplings for \ZZ\ production.
Since they enter linearly in the effective Lagrangian, they will appear
quadratically in the amplitude of the \ZZllll\ process. The differential \cx\
can thus be written as:
\begin{eqnarray}\label{eqn:dsigma}
{\rm d}\sigma_{\rm SM+TGC} & = & F_{00} + f_4^\gamma F_{01} + f_4^Z F_{02} + f_5^\gamma F_{03} + f_5^Z F_{04}  \nonumber \\
&+& \left(f_4^\gamma\right)^2F_{11} + f_4^\gamma f_4^Z F_{12} +  f_4^\gamma f_5^\gamma F_{13} + f_4^\gamma f_5^Z F_{14}  \nonumber \\
&+& \left(f_4^Z\right)^2F_{22} + f_4^Zf_5^\gamma F_{23} + f_4^Zf_5^Z F_{24}  \nonumber \\
&+& \left(f_5^\gamma\right)^2F_{33} + f_5^\gamma f_5^Z F_{34} \nonumber \\
&+& \left(f_5^Z\right)^2F_{44}
\end{eqnarray}
where $F_{ij}$ are coefficients. A priori, there are 25 different coefficients. However, using the 
symmetry property of the coefficients ($F_{ij}=F_{ji}$), it is seen 
that only $25-10=15$ are independent.$F_{00}$ corresponds to the contribution of \sm
diagrams and the rest consist of operator contributions associated with the
anomalous couplings. Using this expression, an event generated assuming only
\sm\ couplings can be assigned a weight corresponding to the differential \cx\ assuming TGC couplings
at some specific value of the four couplings:
\begin{equation}
{\rm weight} = \frac{{\rm d}\sigma_{\rm SM+TGC}}{{\rm d}\sigma_{\rm SM}}
\end{equation}
By assigning such weights to every event in the sample it is possible to reweight a sample
generated with only \sm\ couplings to a sample with the TGC couplings.
This procedure can be easily extended to reweight any sample generated assuming
a given set of TGCs to any other set of TGCs. For example, to reweight a sample
generated assuming only \sm\ couplings to sample assuming \ffourg=0.1, one would
apply to each event a weight:
\begin{eqnarray}\label{revisit}
{\rm weight} = \frac{F_{00} + 0.1 \cdot  F_{01} + 0.1\cdot 0.1 \cdot
F_{11}}{F_{00}}
\end{eqnarray}
The \Fij s are completely
specified by the kinematics of the incoming and outgoing particles, and so must
be evaluated on an event by event basis. They are
independent of the sizes of the anomalous couplings, although they do depend on
the choice of \formfactor. They are determined by using the matrix elements for
\ZZlll\ production in the presenced of TGCs. By using Equation~\ref{eqn:dsigma} it 
is possible to write down 15 equations that uniquely determine the 
coefficients $F_{ij}$ in terms of matrix elements assuming particulare values
for the TGC couplings.
To illustrate the procedure, consider the simplified situation where 
there is just one coupling constant $f$. In this case, there are 3 coefficients 
to be determined:
\begin{eqnarray}\label{revisit2}
{\rm d}\sigma_{\rm SM+TGC} = F_0 + fF_1 + f^2F_2 
\end{eqnarray}
where  $F_0={\rm d}\sigma_{\rm SM}$.

Using three different values of $f$, e.g $f=\{0,1,-1\}$, three independent
equations may be written down:
\begin{eqnarray}\label{matrix1}
\left(\begin{array}{c}
{\rm d}\sigma_1\\
{\rm d}\sigma_2\\
{\rm d}\sigma_3
\end{array}\right) =
\left[\begin{array}{ccc}
1 & 0 & 0 \\
1 & 1 & 1 \\
1 & -1 & 1
\end{array}\right]
\left(\begin{array}{c}
F_0\\
F_1\\
F_2
\end{array}\right)
\end{eqnarray}
Denoting the matrix containing the coupling values $\hat{A}$, the cross sections
d$\vec{\sigma}$ and the coefficients $\vec{F}$, the equations are easily
manipulated to give the coefficients 
\begin{eqnarray}\label{matrix2} {\rm
d}\vec{\sigma}=\hat{A}\vec{F} \qquad \Rightarrow \qquad \vec{F}=\hat{A}^{-1}{\rm
d}\vec{\sigma} 
\end{eqnarray} 
Clearly, $\hat{A}$ must be invertible. This is the
case if the couplings are chosen such that the three equations in~\ref{matrix1}
are independent. When considering all four couplings at the same time, the
matrix $\hat{A}$ is 15$\times$15 and ${\rm d}\vec{\sigma}$ and $\vec{F}$ are
15-dimensional vectors.

The TGC matrix elements are obtained from the next-to-leading order \mc\
generator BHO~\cite{bho}. Matrix elements from the leading-order Baur-Rainwater
MC generator~\cite{Baur:1994au} are also used as a cross-check. The matrix
elements are introduced into the \AfterBurner\ framework~\cite{Bella:2008wc},
which enables a calculation of the amplitude given the four-vectors and PDG
codes of the incoming partons and outgoing particles from the hard process. 

Whilst it is possible to reweight a \sm\ sample to a TGC point, this suffers
from a lack of statistics in the tails of distributions such as the \pt\ of the
\Z\ bosons and the four-lepton invariant mass, where the presence of TGCs
enhances the cross section at high momentum and high invariant mass. Instead,
samples generated with values of the TGC couplings set near previously set
experimental limits are used; this ensures statistics in the tails of the
distributions. The \powhegBox\ and \ggZZ\ generators used to generate the nominal signal
samples do not model \TGC s; instead the samples are generated using \sherpa,
which does. For the 7~\tev\ analysis, four samples are used, with couplings
... For the 8~\tev\ analysis, three sample are used.

\subsection{Validation of Reweighting Procedure}

To demonstrate the performance of the reweighting procedure, the  \fig{} shows

\subsection{Yield Coefficients}

\section{Bin Optimisation}

\section{Expected Limits}

\section{Observed Limits}
