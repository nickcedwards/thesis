\graphicspath{{Chapters/TheoryZZProduction/Figures/}}
\chapter{ZZ Production}
\label{chap:TheoryZZProduction}

\section{Introduction}

\begin{figure}
\centering


\mbox{
	\subfigure[]{
                \begin{fmffile}{gluonbox}
                \begin{fmfgraph*}(40,30)
                \fmfleft{i1,i2}
                \fmflabel{$g$}{i1}
                \fmflabel{$g$}{i2}
                \fmfright{o1,o2}
                \fmflabel{$Z$}{o1}
                \fmflabel{$Z$}{o2}
                \fmf{gluon}{i1,v1} %incoming gluon
                \fmf{fermion}{v1,v3}% + bottom of box --> correct
                \fmf{photon}{v3,o1}
                \fmf{photon}{v4,o2}
                \fmf{fermion}{v4,v2}% + top of box --> correct
                \fmf{gluon}{v2,i2}
                \fmf{fermion}{v2,v1}% + left hand side of box correct
                \fmf{fermion}{v3,v4} % + right hand side of box
                % uncommment this line if you want dots at the vertices
                %\fmfdotn{v}{4}
                \end{fmfgraph*}
                \end{fmffile}
\vspace{20mm}
                }
        }

                \mbox{
	\subfigure[]{
                \begin{fmffile}{tchan}
                \begin{fmfgraph*}(40,30)
                \fmfleft{i1,i2}
                \fmfright{o1,o2}
                \fmflabel{$u,d$}{i2}
                \fmflabel{$\bar{u},\bar{d}$}{i1}
                \fmfright{o1,o2}
                \fmflabel{$Z/\gamma^{*}$}{o1}
                \fmflabel{$Z/\gamma^{*}$}{o2}
                \fmf{fermion}{i2,v2,v1,i1}
                %\fmf{fermion}{v2,i2}
                %\fmf{fermion}{v1,i1}
                \fmf{photon}{v1,o1}
                \fmf{photon}{v2,o2}
                % uncommment this line if you want dots at the vertices
                %\fmfdotn{v}{4}
                \end{fmfgraph*}
                \end{fmffile}
\vspace{20mm}
                }
        }


\caption{A diagram!}
\end{figure}

%Production of pairs of \Z\ bosons, so called `diboson \ZZ\ production' is an
%extremely rare process at particle colliders, but has a very striking signature
%and low backgrounds. The study of diboson \ZZ\ production is of great interest
%as it provides a unique opportunity to probe the electroweak sector of the
%Standard Model. For example, the $ZZZ$ and $ZZ\gamma$ neutral triple gauge boson
%couplings (nTGCs) are zero in the Standard Model, but are predicted to be at the
%level of $10^{-4}$ to $10^{-3}$ in certain new-physics
%models~\cite{Ellison:1998}. Non-zero nTGCs would manifest as an increased \ZZ\
%production cross section, especially at high \ZZ\ invarient mass and transverse
%momentum~\cite{Baur:2000ae}. Standard Model \ZZ\ production is also the irreducible background to $H \rightarrow ZZ$
%decays, one of the key search channels for the Higgs boson at the LHC~\cite{HZZ4l_5fb:CERN-PH-EP-2012-014}.
%
%At a hadron collider, the main process contributing to ZZ
%production is t-channel $q\bar{q} \rightarrow ZZ$ anhiliation, as shown
%in figure~\ref{fig:qqZZdiagram}. Gluon-gluon fusion processes will also
%contribute via box diagrams. Although these are suppressed by a factor of
%$\alpha_s^2$, due to the high gluon content of the proton at LHC energies they
%still contribute approximately 7\%~\cite{Campbell:2011} of the cross-section. 
%
%%The theoretical cross-section predictions 
%%for the ZZ production processes are shown in table~\ref{table:zzcx}.
%
%\begin{figure}[htbp]
%  \begin{center}
%  \includegraphics[width=0.9\textwidth]{figures/zzdiagram.png}
%  \icaption{Tree-level Feynman diagrams for \ZZ\ production through the \qqbar\ initial state in 
%           hadron colliders. The $s$-channel diagram, on the right, contains the $ZZZ$ 
%           ($ZZ\gamma$) neutral triple gauge boson coupling vertex, which is zero in the Standard 
%           Model.}
%\label{fig:LOdiagrams}
%\end{center}
%\end{figure}
%
%Diboson ZZ production was first observed in $e^+e^-$ collisions at LEP in 1997.
%The L3 experiment published the first observation and cross section measurement
%of on-shell \ZZ\ production~\cite{Acciarri1999281}. They analysed 55.3 \ivpb\
%of data collected at an average centre-of-mass energy of 182.7 GeV. Separate
%selections were applied in the various decay channels. Whilst no events were
%observed in the $\ell\ell\ell\ell$ final states, a total of 63 were observed in
%the other visible final states. The majority (47) of these were in the
%all-hadronic channel, which suffered from high backgrounds from $e^+e-
%\rightarrow qq \gamma$ and $e^+e- \rightarrow WW$. In this channel a neural
%network method was used to distinguish the signal events from the background. A
%log-likelihood fit of the neural network output and the observed mass spectra in
%the other channels was used to combine the channels, and gave a cross section of
%$\sigma_{ZZ} = 0.30^{+0.22 +0.07}_{-0.16 -0.03}$, in very good agreement with
%the standard model prediction. The data were fitted to determine the magnitude
%of the anomalous triple gauge couplings. These were found to be consistent with
%the SM predictions of 0 to 95\% confidence level. Updated cross section
%measurements have since been published by L3 and the other 3 LEP experiments, in
%all cases measuring cross sections for \ZZ\ production consistent with the
%standard model, and finding no evidence for anomalous couplings.
%
%Measurements of the \ZZ\ cross section have also been made in $p \bar{p}$
%collisions at a centre of mass energy of $\sqrt{s} = 1.96$ \tev\ at the Tevatron, at
%both the D0 and the CDF experiments. The most recent D0 publication reports the
%observation of 10 candidate \ZZ\ events with an expected background of $0.37
%\pm 0.13$. The measured cross section was $1.33^{+0.5}_{-0.4}$ (stat) $\pm$ 0.12
%(syst) $\pm$ 0.09 (lumi) pb, with a signal significance of greater than
%$6\sigma$. The result is consistent with the standard model prediction of 1.4
%$\pm$ 0.1 pb. 
%
%The cross section at the LHC for a centre-of-mass energy of 7 \TeV\ is predicted to be $\sim$5 times larger 
%than at the Tevatron. 
%Recently the CMS experiment have measured the \ZZ\ production cross
%section~\cite{CMS-PAS-EWK-11-010} and found it consistent with the Standard Model prediction.
%
%
%%%%%%%%%%% With ZZ*
%%In this note we present three measurements of the \ZZ\footnote{Throughout this paper \Z\ should be taken 
%%to mean $\Z/\gamma^{*}$.} production cross section 
%%in proton-proton collisions at a centre-of-mass energy $\sqrt{s}$ of 7 \TeV:
%%two ``fiducial cross-sections'', measured in different dilepton mass regions,
%%and a total cross section. 
%
%In this report I describe a measurement of the \ZZ\ production cross section
%(where \Z\ should be taken 
%to mean $\Z/\gamma^{*}$) in proton-proton collisions at a centre-of-mass energy $\sqrt{s}$ of 7 \TeV.
%A ``fiducial cross section'' is measured in a phase-space that closely matches
%to the experimental selection, 
%thus reducing the theory-dependent systematic uncertainties of the measurement.
%This is then extrapolated to a total cross-section using corrections taken from
%Monte-Carlo simulation, correcting for the well known $\Z \rightarrow \ell \ell$
%branching fractions.
%The measurement uses a data sample corresponding to an integrated luminosity of 4.7 \ifb\ collected by the ATLAS detector at 
%the LHC during 2011. 
%The cross section for on-shell \ZZ\ production is
%predicted at next-to-leading order in QCD (NLO), using the MSTW2008~\cite{bib:MSTW2008} NLO parton density function (PDF) set, 
%to be $6.5^{+0.3}_{-0.2}\ \pb$ \cite{Campbell:2011}.  
%
%Candidate \ZZ\ events are reconstructed in the $\ZZ\rightarrow\ll\ll$ decay
%channel, where $\ell$ can be an electron or a muon, giving rise to three decay
%channels: \eeee, \mumumumu\ and \eemumu.
%Although \zzllll\ decays constitute only $\sim$0.5\% of the total on-shell \ZZ\ cross section, 
%the experimental signature of four high transverse-momentum, isolated leptons is very clean.  
%
%A common fiducial phase space defintion is used for all three decay channels (\ee\ee\,
%\mumu\mumu\ and \ee\mumu), and is defined as follows:
%
%\begin{itemize}
%\item $(\Z/\gamma^*)(\Z/\gamma^*)\rightarrow\ll\ll$, $\ell = e,\mu$;
%\item $66 < m_{12}(\Z/\gamma^*) <  116\GeV$;
%\item $66 < m_{34}(\Z/\gamma^*) <  116\GeV$;
%\item $\pT^{\ell} > 7\GeV$;
%\item $|\eta^{\ell}| < 2.7$.
%%\item $|\eta^{\ell}| < 3.16$.
%%\item $\mathrm{min}(|\Delta R(\ell,\ell)|) < 0.2$.
%\end{itemize}
%
%%The background from $\ZZ\ra\ll\tautau$ where
%%the tau leptons decay to electrons or muons is less than 2.5\% and is taken into account in the correction factor translating the reconstructed event count to the number of true   $\ZZ\rightarrow\ll\ll$ decays within the fiducial phase space. 
%
%The fiducial cross section is obtained from the reconstructed number of events
%using a correction factor estimated using Monte-Carlo to extrapolate from the
%number of events to the true number of events in the fiducial volume. The
%correction factors are different for each of the three decay channels reflecting
%not only the different efficiencies, but also the slightly different event
%selection amongst the three channels.
%
%The total \ZZ\ cross section
%is obtained then from the \ZZ\ fiducial cross section 
%using the $\Z\rightarrow\ll$ branching ratio~\cite{PDG} and a correction factor 
%for the kinematic and geometrical acceptance, 
%calculated using the MCFM program~\cite{Campbell:2011}.
%In calculating the invariant masses of the lepton pairs at generator level, photons within 
%$\Delta R \equiv \sqrt{\Delta \phi^2 + \Delta \eta^2} = 0.1$ of the lepton are included in the lepton four-momentum.    
%
%
%\section{Standard Model ZZ Production}
%\subsection{Feynman Diagrams}
%\subsection{Fiducial Volume Defintion}
%
%\subsection{\zzllll\ Fiducial Cross Section Definitions}
%
%The \zzllll\ on-shell (\ZZ) fiducial cross section is defined as:
%\begin{itemize}
%\item{\ZorgZorglplmlplm, $\ell = e,\mu$}
%\item{ $66 < m_{12}(\Zorgv) <  116\GeV$, where $m_{12}(\Zorgv)$ is
%the mass of the \Z\ reconstructed from the first and second leptons.  The
%lepton pairings are assigned by choosing the set of 
%same-flavor, opposite-sign lepton pairs that minimises the sum of distances from
%the PDG value of the \Z\ mass:
%\begin{equation}
%|m_{1,2}(\Zorgv) - \mZPDG| + |m_{3,4}(\Zorgv) - \mZPDG|
%\end{equation}
%}
%\item{ $66 < m_{34}(\Z/\gamma^*) <  116\GeV$, where $m_{34}(\Z/\gamma^*)$ is
%the mass of the \Z\ reconstructed from the third and fourth leptons, and the
%lepton pairing is done as mentioned above;}
%\item $\pT^{\ell} > 7\GeV$;
%\item $|\eta^{\ell}| < 3.16$.
%\item{$\mathrm{min}(\Delta R(\ell,\ell)) > 0.2$.}
%\end{itemize}
%
%The \zzllll\ fiducial cross-section, allowing one \Z\ to be off-shell ($ZZ^*$), is defined as:
%
%\begin{itemize}
%\item $(\Z/\gamma^*)(\Z/\gamma^*)\rightarrow\ll\ll$, $\ell = e,\mu$;
%\item $66 < m_{12}(\Z/\gamma^*) <  116\GeV$;
%\item $m_{34}(\Z/\gamma^*) > 20\GeV$;
%\item $\pT^{\ell} > 7\GeV$;
%\item $|\eta^{\ell}| < 3.16$.
%\item $\mathrm{min}(|\Delta R(\ell,\ell)|) > 0.2$.
%\end{itemize}
%
%The $\Delta R$ cut requires the minimum $\Delta R$ between any two out of the
%four selected leptons in the event to be greater than 0.2.
%
%Cross-sections for the $ee\mu\mu$ final state with a tight mass cut corresponding to $ZZ$ production are  shown in Table~\ref{tab:MCFM3}.
%%, together with the
%%fraction of the cross section in the fiducial volume $A_{ZZ}$.
%%For complete systematic uncertainties on the $A_{ZZ}$ acceptance see Section~\ref{subsec:AccVal}.
%Those with a loose mass cut allowing one \Z\ to be off-shell, are shown in Table~\ref{tab:MCFM4}.  
\subsection{Cross Section}
\subsection{Generator level distributions}
\section{Anomolous Triple Gauge Couplings}
\section{ZZ Resonances}
