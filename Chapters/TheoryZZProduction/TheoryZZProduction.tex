\graphicspath{{Chapters/TheoryZZProduction/Figures/}}
\chapter{ZZ Production}
\label{chap:TheoryZZProduction}

\section{Introduction}
\section{Standard Model ZZ Production}
\subsection{Feynman Diagrams}
\subsection{Fiducial Volume Defintion}

\subsection{\zzllll\ Fiducial Cross Section Definitions}

The \zzllll\ on-shell (\ZZ) fiducial cross section is defined as:
\begin{itemize}
\item{\ZorgZorglplmlplm, $\ell = e,\mu$,
where each \Z\ decays to a particle-antiparticle pair of a given lepton flavor,
i.e. $\Z\rightarrow e^{-}e^{+}$ or $\Z\rightarrow \mu^{-}\mu^{+}$;}
\item{ $66 < m_{12}(\Z/\gamma^*) <  116\GeV$, where $m_{12}(\Z/\gamma^*)$ is
the mass of the \Z\ reconstructed from the first and second leptons.  The
same-flavor, opposite-sign lepton pairings are done such that the mass of the 
reconstructed \Z\ is closest to the PDG value of the \Z\ mass;}
\item{ $66 < m_{34}(\Z/\gamma^*) <  116\GeV$, where $m_{34}(\Z/\gamma^*)$ is
the mass of the \Z\ reconstructed from the third and fourth leptons, and the
lepton pairing is done as mentioned above;}
\item $\pT^{\ell} > 7\GeV$;
\item $|\eta^{\ell}| < 3.16$.
\item{$\mathrm{min}(\Delta R(\ell,\ell)) > 0.2$.}
\end{itemize}

The \zzllll\ fiducial cross-section, allowing one \Z\ to be off-shell ($ZZ^*$), is defined as:

\begin{itemize}
\item $(\Z/\gamma^*)(\Z/\gamma^*)\rightarrow\ll\ll$, $\ell = e,\mu$;
\item $66 < m_{12}(\Z/\gamma^*) <  116\GeV$;
\item $m_{34}(\Z/\gamma^*) > 20\GeV$;
\item $\pT^{\ell} > 7\GeV$;
\item $|\eta^{\ell}| < 3.16$.
\item $\mathrm{min}(|\Delta R(\ell,\ell)|) > 0.2$.
\end{itemize}

The $\Delta R$ cut requires the minimum $\Delta R$ between any two out of the
four selected leptons in the event to be greater than 0.2.

Cross-sections for the $ee\mu\mu$ final state with a tight mass cut corresponding to $ZZ$ production are  shown in Table~\ref{tab:MCFM3}.
%, together with the
%fraction of the cross section in the fiducial volume $A_{ZZ}$.
%For complete systematic uncertainties on the $A_{ZZ}$ acceptance see Section~\ref{subsec:AccVal}.
Those with a loose mass cut allowing one \Z\ to be off-shell, are shown in Table~\ref{tab:MCFM4}.  
\subsection{Cross Section}
\subsection{Generator level distributions}
\section{Anomolous Triple Gauge Couplings}
\section{ZZ Resonances}
